\documentclass[12pt,a4paper]{report}

% Paquetes para el manejo de la lengua y la codificación de los caracteres
\usepackage[spanish]{babel} % Para tener los elementos de texto en español
\usepackage[T1]{fontenc} % Set the font (output) encodings for español

% Paquetes para el manejo de gráficos e imágenes
\usepackage{graphicx} % Mejoras sobre el paquete graphics
\graphicspath{ {imagenes/} } % carpeta en la que va a buscar por imágenes

% Paquetes para el manejo de matemáticas
\usepackage{amsmath} % para poner flechas dentro del código

% Paquetes para el manejo de la geometría del documento y la justificación del texto
\usepackage[a4paper,top=2cm,bottom=2cm,left=3cm,right=3cm,marginparwidth=1.75cm]{geometry} % para la portada. REVISAR QUE HACE
\usepackage[protrusion=false, expansion=true]{microtype} % mejora la justificación del documento

% Paquetes para el manejo de hipervínculos
\usepackage[pdfpagelabels,colorlinks=true, allcolors=black]{hyperref} %pone los links en negro

% Paquetes para el manejo de la bibliografía
\usepackage[style=ieee]{biblatex}
\usepackage{csquotes} % Facilitar el trabajo con citas
\addbibresource{export.bib}
\setcounter{biburlnumpenalty}{9000}
\setcounter{biburllcpenalty}{9000} 
\setcounter{biburlucpenalty}{9000} 

\addto\extrasspanish{
    \def\sectionautorefname{Capítulo}
    \def\subsectionautorefname{Apartado}
}

% Paquetes para el manejo de código fuente
\usepackage{listingsutf8}
\usepackage{pxfonts,fix-cm}
\usepackage{accsupp} % Para poder poner los número en el código y que no se copien al seleccionar el código
\usepackage{xcolor}
\newcommand{\noncopynumber}[1]{%
    \BeginAccSupp{method=escape,ActualText={}}%
    #1%
    \EndAccSupp{}%
}

% Configuración de la presentación del código
\definecolor{sviolet}{HTML}{6C71C4}
\definecolor{sbase1}{HTML}{93A1A1}
\definecolor{sblue}{HTML}{268BD2}
\definecolor{scyan}{HTML}{2AA198}
\definecolor{sbase00}{HTML}{657B83}
\lstset{
    inputencoding=utf8/latin1,
    language=Java, 
    columns=fullflexible, % si lo comentas queda algo mas bonito
    keepspaces=true, % si lo comentas queda algo mas bonito
    sensitive=true,
    aboveskip=\baselineskip,
    belowskip=\baselineskip,
    frame=lines,
    xleftmargin=\parindent,
    belowcaptionskip=1\baselineskip,
    basicstyle=\color{sbase00}\ttfamily,
    keywordstyle=\color{scyan},
    commentstyle=\color{sbase1},
    stringstyle=\color{sblue},
    numberstyle=\color{sviolet},
    identifierstyle=\color{sbase00},
    breaklines=true,
    showstringspaces=false,
    tabsize=2
}

\lstdefinestyle{mystyle}{
    basicstyle=\ttfamily\footnotesize,
    breaklines=true,
    columns=fullflexible,
    frame=single,
    captionpos=b,
    keepspaces=true,
    showspaces=false,
    showstringspaces=false,
    numberstyle=\tiny,
    numbers=left,
    stepnumber=1,
    numbersep=5pt
}

% Paquetes para el manejo de la disposición de los elementos en la página
\usepackage{float} % Para centrar las imágenes 
\usepackage{parskip} % si no pones esto, el titulo de la portada sale mal
% \usepackage[all]{hypcap} % Ajusta los enlaces para que apunten a la parte superior de las figuras y tablas
\usepackage[hypcap=true]{caption} % hypcap esta desactualizado
\usepackage{chngcntr} % Desvincula el contador de las figuras del contador de las secciones
\counterwithout{figure}{chapter} % para que la numeración de las figuras no siga la de la sección en la que esta
\usepackage{awesomebox} % Bloques de texto más molones
\setlength{\parskip}{0pt} % definimos la separación entre párrafos
\setlength{\parindent}{20pt} % definimos la identación de los párrafos

% Paquetes para la creación de la portada
\usepackage{tikz} % para crear gráficos y figuras mas complejos. figura azul de portada
\usepackage{changepage} % para editar margenes y otras dimensiones de la pagina
\usepackage{afterpage}% for "\afterpage"

\definecolor{color_29791}{rgb}{0,0,0} % color para la figura de la portada
\definecolor{color_104998}{rgb}{0.290196,0.494118,0.733333} % color para la figura de la portada
\definecolor{AzulCeleste}{RGB}{31, 130, 192} % para la portada


\usepackage{booktabs}
\usepackage{array}
\usepackage{tabularx}
\usepackage{todonotes}

% Paquetes deshabilitados
% \usepackage{pict2e} % mejora del paquete picture. Para cosas mas complejas usar tikz
% \usepackage{wasysym} % da conflictos con amsmath
% \usepackage{latexsym} 
% \usepackage[utf8]{inputenc} % ya no hace falta desde 2018
% \usepackage{pdfpages} % Para insertar pdf en el documento
% \usepackage[none]{hyphenat} %evitamos que rompa las palabras

% Comando para escribir PFG
\newcommand{\pfg}{Proyecto Fin de Grado }

\title{Sistema multiplataforma en Rust para Análisis 3D 
Biomecánico con Motion Capture}
\author{Rubén Agustín González}

\begin{document}

% Primera hoja (obligatorio)
\hypersetup{pageanchor=false}
\begin{titlepage}
    \thispagestyle{empty}
    \newgeometry{margin=0in}
\begin{tikzpicture}[overlay]\path(0pt,0pt);\end{tikzpicture}
\begin{picture}(-5,0)(2.5,0)
	\put(38.88,-35.76001){\fontsize{12}{1}\usefont{T1}{ptm}{m}{n}\selectfont\color{color_29791} }
	\put(38.88,-49.56){\fontsize{12}{1}\usefont{T1}{ptm}{m}{n}\selectfont\color{color_29791} }
	\put(38.88,-792.24){\fontsize{12}{1}\usefont{T1}{ptm}{m}{n}\selectfont\color{color_29791} }
	\put(178.5,-135.7297){\includegraphics[width=252.1922pt,height=78.8pt]{latexImage_b33e2d3a1e077c187ed4eaa1710c29b2.png}}
\end{picture}
\begin{tikzpicture}[overlay]
	\path(0pt,0pt);
	\draw[color_104998,line width=15pt,line join=round]
	(38.8501pt, -33.38pt) -- (538.5001pt, -32.33002pt)
	;
	\draw[color_104998,line width=15pt,line join=round]
	(36.5501pt, -787.9316pt) -- (505.8501pt, -787.9316pt)
	;
	\draw[color_104998,line width=15pt,line join=round]
	(46.2502pt, -25.97992pt) -- (44.1502pt, -784.6299pt)
	;
	\draw[color_104998,line width=15pt,line join=round]
	(520.1499pt, -788.7799pt) -- (538.3999pt, -788.7799pt)
	;
\end{tikzpicture}

\tikz[remember picture,overlay]
	\node[opacity=0.3, inner sep=0pt] at ([xshift=2cm, yshift=-6cm]current page.center)
    {\includegraphics[width=0.5\paperwidth, height=0.5\paperheight, keepaspectratio]{escudotrasparente.jpg}};



\vspace{7cm}
\begin{center}
	{\fontsize{20}{24} \textbf{PROYECTO FIN DE GRADO} } \\
	
\end{center}
\vspace{1,3cm}
%\begin{spacing}{2}
\begin{adjustwidth}{4cm}{3cm}
	{\setlength{\parskip}{0pt} \setlength{\parindent}{0pt}

{\fontsize{11}{13,2}
	{\textbf{TÍTULO:} Sistema multiplataforma en Rust para Análisis 3D Biomecánico con Motion Capture}\vspace{19pt}
	
	{\textbf{AUTOR/A:} Rubén Agustín González }\vspace{19pt}
	
	{\textbf{TITULACIÓN:} Doble grado en Ingeniería electrónica de comunicaciones e Ingeniería telemática }\vspace{19pt}
	
	{\textbf{DIRECTOR/A:} Enrique Navarro Cabello } \vspace{19pt}
	
	{\textbf{TUTOR/A:} Juana María Gutiérrez Arriola} \vspace{19pt}
	
	{\textbf{DEPARTAMENTO:} DTE}\vspace{19pt}
	
	\hspace*{\fill}	{\textbf{VºBº TUTOR/A}  }\vspace{17pt}
	
	{\textbf{Miembros del Tribunal Calificador:}}\vspace{19pt}
	
	{\textbf{PRESIDENTE/A:} Rafael José Hernández Heredero}\vspace{19pt}
	
	{\textbf{TUTOR/A:} Juana María Gutiérrez Arriola}\vspace{19pt}
	
	{\textbf{SECRETARIO/A:} Martina Eckert}\vspace{19pt}
	
	{\textbf{Fecha de lectura:} 3 de junio de 2025}\vspace{19pt}
	
	{\textbf{Calificación:} }\vspace{19pt}
	
	\hspace*{\fill}{\textbf{El Secretario/La Secretaria,}}\vspace{19pt}
}
}
\end{adjustwidth}

%\end{spacing}
\restoregeometry
\end{titlepage}
\pagenumbering{Roman}

\listoftodos

\newpage
% Resumen (obligatorio)
\begin{abstract}
    Este proyecto es el resultado de una colaboración entre la facultad de \acs{INEF} y el centro de investigación \acs{GAMMA}, ambos pertenecientes a la \acs{UPM}. Consiste en el desarrollo de una aplicación para la visualización de movimientos deportivos capturados mediante un sistema de Motion Capture, en el formato de archivos \acs{C3D}. 

La aplicación permite reproducir un fichero sobre un entorno tridimensional, pudiendo visualizar el movimiento de manera interactiva. El programa admite un fichero de configuración para personalizar la visualización, pudiendo variar características de los marcadores. Además, es posible añadir uniones entre marcadores y vectores de velocidad y aceleración. 

Esta aplicación se ha desarrollado íntegramente en Rust, utilizando únicamente herramientas de este lenguaje.
\end{abstract}
\renewcommand{\abstractname}{Abstract} % para poner el nombre que queramos
\newpage
% Abstract (obligatorio)
\vspace{10cm}
\begin{abstract}
    This project is the result of a collaboration between the \acs{INEF} school and the \acs{GAMMA} investigation center at the \acs{UPM}. This application aims to develop a tool for visualizing sports movements captured through a Motion Capture system, in the \acs{C3D} file format. Additionally, the application provides features for interactive visualization and customization of marker characteristics, enhancing the analysis of captured movements.

This application has been developed in pure Rust.

\end{abstract}

% Agradecimientos
\chapter*{Agradecimientos}
\noindent Añade aqui tus agradecimientos...

% Índice de contenidos (obligatorio)
\hypersetup{pageanchor=false}
\hypersetup{linkcolor=black}
\tableofcontents

% Índice de figuras
\listoffigures
\listoftables

\chapter*{Acrónimos}

% Lista de acrónimos: Añadir después a configuración de espanso para sustitución automática
\begin{acronym}
    \acro{C3D}{\textit{Coordinate 3D}}
    \acro{DOM}{\textit{Document Object Model}}
    \acro{ECS}{\textit{Entities, Components, Systems}}
    \acro{INEF}{Instituto Nacional de Educación Física}
    \acro{JSON}{\textit{JavaScript Object Notation}}
    \acro{LBHD}{\textit{Left Back Head}}
    \acro{LFHD}{\textit{Left Front Head}}
    \acro{MoCap}{\textit{Motion Capture}}
    \acro{RBHD}{\textit{Right Back Head}}
    \acro{RFHD}{\textit{Right Front Head}}
    \acro{RFIN}{\textit{Right Finger}}
    \acro{RFRA}{\textit{Right Forearm}}
    \acro{TOML}{\textit{Tom's Obvious, Minimal Language}}
    \acro{UPM}{Universidad Politécnica de Madrid}
    \acro{YAML}{\textit{YAML Ain't Markup Language}}
\end{acronym}



% Contenidos
\pagenumbering{arabic} % cambiamos la numeración de las páginas

\chapter{Introducción}
\label{sec:cap1}
\section{Marco y motivación del proyecto}

\noindent La introducción ha de hacer una resumida descripción del marco tecnológico donde se ubica el proyecto, justificando su necesidad y los objetivos que se buscan con su realización, es decir, del problema que se pretende resolver o analizar. También se puede incluir una visión general de las soluciones propuestas, sin entrar en detalles. La introducción debe concluir con una breve descripción (de no más de una página) de la estructura del informe del proyecto, resumiendo de manera escueta los contenidos de cada capítulo. Es frecuente que la introducción sea el último apartado del informe que se escriba.

Las figuras deben citarse en el texto al menos una vez antes de la aparición de cada una. Ver como ejemplo la \autoref{fig:esquema-internet}, en la que se observa un esquema bla, bla, bla, bla...

\begin{figure}[H]
    \centering
    \includegraphics[width=\textwidth]{Imagen 1.png}
    \caption{Esquema conceptual de internet}
    \label{fig:esquema-internet}
\end{figure}

% Las referencias se deben citar asi\cite{javaccgithub}.
\section{Objetivos técnicos y académicos}

\noindent Los objetivos de este proyecto fin de carrera son, desde el punto de vista técnico:

\begin{itemize}
    \item Bla bla bla
\end{itemize}

Desde el punto de vista académico, el proyectista adquiere las siguientes competencias y habilidades:

\begin{itemize}
    \item Bla bla bla
\end{itemize}

\section{Estructura del resto de la memoria}

\noindent En los próximos capítulos de la memoria se ofrecen información sobre... (no más de una página).

\chapter{Estado del arte} \label{sec:cap2}

\noindent El estudio del movimiento del cuerpo humano es una tarea compleja, que requiere de amplios conocimientos en el sector de la biomecánica, y herramientas específicas para su análisis. En este sentido, el sistema de captura de movimiento, conocido como \ac{MoCap}, es una herramienta fundamental para el estudio del movimiento humano \autocite{taiQueEsMotion2024}.

Durante este capítulo se presenta el estado del arte de la tecnología de captura de movimiento, y se analizan las diferentes herramientas que existen en el mercado para la visualización de los datos obtenidos. A continuación, se presenta el formato de fichero \ac{C3D}, que es el formato utilizado para almacenar los datos obtenidos por el sistema de captura de movimiento. Por último, se presentan los diferentes lenguajes de programación valorados para el desarrollo del proyecto, y se justifica la elección del mismo.

\section{Sistemas \textit{Motion Capture}}
El \textit{Motion Capture} es una técnica que permite capturar el movimiento de un objeto en tiempo real, y representarlo en un entorno virtual. Esta técnica se utiliza en diferentes campos, como la animación, la medicina, la biomecánica o la robótica; y se basa en la captura de la posición de diferentes marcadores en el espacio. Para lograr la captura, existen diferentes tecnologías, como cámaras de vídeo, sensores inerciales, sensores de ultrasonidos, etc. \autocite{taiQueEsMotion2024}

\subsection{\textit{Hardware} disponible}
En la facultad de \ac{INEF} de la \ac{UPM}, se utiliza un sistema de Motion Capture basado en cámaras de infrarrojo, que utilizan para capturar movimientos deportivos\footnote{La facultad de INEF cuenta con 6 cámaras infrarrojas de la marca Vicon, capaces de capturar fotogramas a una frecuencia de hasta 500 \ac{Hz} \autocite{FacultadCienciasActividad}}. Estos marcadores se colocan en diferentes partes del cuerpo del deportista, logrando capturar un movimiento preciso. Los datos capturados se almacenan en un fichero en formato \ac{C3D}, que contiene la información necesaria para la representación del movimiento.

Las tecnologías que existen en el mercado para la visualización de estos movimientos son muy costosas, y no siempre se adaptan a las necesidades de los investigadores. Por ello, en este proyecto se ha desarrollado una aplicación para la visualización de movimientos capturados en formato \ac{C3D}. La aplicación permite la visualización de los movimientos en un entorno tridimensional, pudiendo variar diferentes parámetros de la visualización, como la velocidad de reproducción, la posición de la cámara, la representación de los marcadores, de las uniones, vectores o fuerzas.

Es relevante señalar que este Proyecto se ciñe al sistema \ac{MoCap} disponible en la facultad de INEF de la \ac{UPM}, por el elevado coste de estos sistemas y la imposibilidad de trabajar con alternativas de la competencia. Sin embargo, existen otras alternativas con gran aceptación en el mercado.

\subsection{Alternativas}
Existen diferentes alternativas a los sistemas Vicon disponibles en el mercado, como los sistemas OptiTrack o Qualisys. En el artículo \autocite{article} se comparan estos sistemas, quedando patente que estos sistemas son capaces de capturar correctamente el movimiento. Además, en la \autoref{fig:errores-mocap} se aprecia como los errores de estos sistemas se han ido reduciendo a lo largo del tiempo.

\begin{figure}[H]
    \centering
    \includegraphics[width=\textwidth]{errores-mocap.png}
    \caption{Error absoluto de diferentes sistemas. Obtenido de \autocite{article}.}
    \label{fig:errores-mocap}
\end{figure}

\section{Ficheros \acs{C3D}} \label{sec:ficheros-c3d}
Entender el formato de los ficheros \ac{C3D} es fundamental para el desarrollo de esta aplicación. 

El formato \ac{C3D} es un formato de fichero binario de dominio público que se crea a mediados de los años 80, en el \textit{National Institutes of Health Biomechanics Laboratory}, en Maryland, USA, y es compatible con todos los principales sistemas de captura de movimiento 3D. Es usado por empresas de las industrias de biomecánica, captura de movimiento y animación \autocite{C3DORGBiomechanicsStandard}.

Un fichero \ac{C3D} es un fichero binario que contiene toda la información de la captura de movimiento. Como se explica en \autocite{C3DORGBiomechanicsStandard}, este fichero se compone de diferentes secciones, que contienen la información de los marcadores para cada fotograma de la captura. Se utilizan estos ficheros en vez de ficheros de texto para reducir el tamaño del fichero, y para garantizar la integridad de los datos.

Es importante señalar que el formato \ac{C3D} contiene información de objetos con coordenadas tridimensionales, pero estos objetos no tienen por qué ser marcadores. Por ejemplo, en el fichero \ac{C3D} se pueden almacenar diferentes tipos de objetos, como marcadores, vectores, fuerzas, etc. Por tanto, es importante entender el formato \ac{C3D} para poder extraer la información necesaria para la representación del movimiento. Además, es necesaria una herramienta para filtrar la información que se quiere representar, ya que un fichero \ac{C3D} puede contener miles de objetos, y no todos ellos son relevantes para la representación del movimiento.

Existen diferentes \textit{parsers} que permiten la lectura de estos ficheros, y la extracción de la información necesaria para la representación del movimiento. En este proyecto se ha modificado el \textit{crate} \texttt{c3dio}, que permite la lectura de ficheros \ac{C3D} en Rust, y tiene un \textit{plugin} para \texttt{Bevy} que facilita la integración con el motor de videojuegos, que se explica en el \autoref{subsec:bevy}. Adicionalmente, con la herramienta explicada en el \autoref{apx:c3d_latex_py}, se puede convertir un fragmento de un fichero \ac{C3D} a un formato leíble por humanos. Se muestra en la \autoref{tab:c3d_data} un fragmento de un fichero \ac{C3D} utilizado durante el desarrollo de la aplicación, que contiene información sobre los primeros diez fotogramas de la captura de movimiento, para dos marcadores, \ac{RFIN} y \ac{RFRA}.

\begin{table}[htbp]
    \centering
    \setlength{\tabcolsep}{5pt}
    \renewcommand{\arraystretch}{1.2}
    \rowcolors{3}{white}{gray!10}
    \begin{tabular}{r|S[table-format=-3.2]S[table-format=-3.2]S[table-format=-3.2]|S[table-format=-3.2]S[table-format=-3.2]S[table-format=-3.2]}
    \toprule
    \multirow{2}{*}{\textbf{Frame}} & \multicolumn{3}{c|}{RFIN                          } & \multicolumn{3}{c}{RFRA                          } \\
      & {\textbf{x}} & {\textbf{y}} & {\textbf{z}} & {\textbf{x}} & {\textbf{y}} & {\textbf{z}} \\
    \midrule
    1 & 46.80 & 3.25 & 664.41 & -23.14 & -4.06 & 827.33 \\
    2 & 46.80 & 3.25 & 664.41 & -23.14 & -4.07 & 827.30 \\
    3 & 46.80 & 3.25 & 664.41 & -23.14 & -4.08 & 827.30 \\
    4 & 46.80 & 3.25 & 664.41 & -23.14 & -4.07 & 827.30 \\
    5 & 46.80 & 3.25 & 664.41 & -23.15 & -4.06 & 827.31 \\
    \midrule
    6 & 46.80 & 3.25 & 664.41 & -23.16 & -4.03 & 827.33 \\
    7 & 46.81 & 3.25 & 664.41 & -23.18 & -4.00 & 827.35 \\
    8 & 46.82 & 3.28 & 664.40 & -23.21 & -3.99 & 827.36 \\
    9 & 46.87 & 3.36 & 664.40 & -23.23 & -4.02 & 827.33 \\
    10 & 46.93 & 3.51 & 664.38 & -23.23 & -4.09 & 827.26 \\
    \bottomrule
    \end{tabular}
    \caption{Fragmento de un fichero \acs{C3D} convertido a tabla}
    \label{tab:c3d_data}
\end{table}
  
Se puede observar que aparecen los dos marcadores mencionados, junto con sus coordenadas en el espacio, representadas por los valores de las columnas \textit{x}, \textit{y} y \textit{z}, en unidades de milímetros.

\subsection{Herramientas actuales para la visualización de ficheros \acs{C3D}}

Las herramientas actuales que existen en el mercado para la visualización de un fichero \ac{C3D} son escasas y generalmente costosas. Una biblioteca famosa en el pasado para el tratamiento de ficheros \ac{C3D} era la \textit{Biomechanics ToolKit} - BTK, pero fue comprada por la compañía canadiense \textit{Movec}, que tiene intención de renovarla bajo el nombre comercial \textit{Bridge} \autocite{Bridge,ProjectBiomechanicalToolKit}. Por otro lado, el entorno de trabajo proporcionado por las cámaras Vicon de \ac{INEF} es antiguo y tosco, no admitiendo características clave en el la época actual, como la visualización en línea. Además, su licencia es propietaria y tiene un coste elevado.

\todo{Imagen del entorno de Enrique}

\section{Lenguaje de programación}

Un aspecto importante para este proyecto es la elección del lenguaje de programación. Al ser un proyecto que no hereda de ningún otro, hay libertad para elegir el lenguaje de programación que se considere más adecuado. Sin embargo, elegir un buen lenguaje de programación es fundamental para el éxito del proyecto, ya que un lenguaje de programación inadecuado puede hacer que el proyecto sea difícil de mantener y de escalar.

Como se explica en el \autoref{sec:ficheros-c3d}, un fichero \ac{C3D} contiene toda la información de la posición de los marcadores a lo largo de la grabación del movimiento. Una frecuencia de muestreo típica en este tipo de ficheros es de 250 \ac{Hz}, lo que significa que se captura la posición de los marcadores 250 veces por segundo. Adicionalmente, al fichero \ac{C3D} se le añaden puntos calculados, que pueden ser articulaciones entre dos marcadores reales, vectores de velocidad, aceleraciones, etc. Por tanto, es común que un fichero \ac{C3D} contenga miles de puntos, lo que hace que la visualización de estos movimientos sea una tarea compleja y exigente en términos de rendimiento.

Adicionalmente, en el planteamiento del proyecto, se especificó que debían existir dos versiones, una de escritorio y una versión web, bajo el mismo código fuente. Esto implica que el mismo código fuente debe poder ser compilado a los sistemas operativos mayoritarios, y además debe poder ser ejecutada en un entorno web. Con estas premisas, se han valorado diferentes lenguajes de programación para el desarrollo de la aplicación.  

\subsection{JavaScript}
JavaScript es un lenguaje de programación interpretado, que se ha convertido en el estándar para el desarrollo de aplicaciones web. Este lenguaje está muy optimizado, gracias a avances en los motores de JavaScript, como SpiderMonkey de Mozilla o V8 de Google \autocite{srinetChromeV8Firefox2022}. Para el desarrollo de aplicaciones de escritorio, JavaScript se puede utilizar con diferentes tecnologías, como Electron \autocite{BuildCrossplatformDesktop}, que permite la creación de aplicaciones de escritorio multiplataforma con tecnologías web. 

Pese a esto, el mayor problema de JavaScript es el rendimiento, ya que este lenguaje no dispone de características de bajo nivel, como la gestión de memoria o la programación concurrente \autocite{MemoryManagementJavaScript2025,pengMultithreadingJavascript2017}. Por tanto, JavaScript no es un lenguaje de programación adecuado para el desarrollo de aplicaciones que requieren un alto rendimiento, como es el caso de la visualización de movimientos en 3D.

\subsection{Python}
Python es un lenguaje de programación interpretado, que se ha convertido en uno de los lenguajes de programación más populares en la actualidad \autocite{TIOBEIndex}. Python es un lenguaje de programación de tipado dinámico, que permite la inferencia de tipos, y que garantiza la seguridad de la memoria en tiempo de ejecución. Este lenguaje de programación resulta muy versátil, utilizándose en diferentes campos, como la inteligencia artificial, el análisis de datos, la programación web, etc. Además, dispone de una gran cantidad de bibliotecas y \textit{frameworks}, que facilitan el desarrollo de aplicaciones de todo tipo.

Sin embargo, al igual que JavaScript, Python no es un lenguaje de programación especialmente rápido, ya que es un lenguaje interpretado, y generalmente sus bibliotecas son pesadas y poco eficientes, por lo que se suele considerar un lenguaje ideal únicamente para prototipado \autocite{SlowestProgrammingLanguages2020}.

\subsection{C++}
C++ es un lenguaje de programación de propósito general, que se utiliza en diferentes campos, como la programación de sistemas, la programación de videojuegos, la programación de aplicaciones de escritorio, etc. C++ es un lenguaje de programación de tipado estático, que permite la inferencia de tipos, y que garantiza la seguridad de la memoria en tiempo de compilación. C++ es un lenguaje de programación muy eficiente, que permite la programación de bajo nivel, y que dispone de una gran cantidad de bibliotecas y frameworks, que facilitan el desarrollo de aplicaciones de todo tipo. C++ es un lenguaje de programación muy rápido, que se utiliza en aplicaciones que requieren un alto rendimiento, como videojuegos, motores de renderizado o sistemas operativos, entre otros. 

Sin embargo, C++ es un lenguaje de programación antiguo, que no dispone de características modernas, como un gestor de paquetes para la gestión de dependencias. Además, C++ no es un lenguaje de programación concebido para el desarrollo de aplicaciones web, y si bien existen herramientas para compilar C++ a \textit{WebAssembly}, su uso no es mayoritario.

\subsection{Rust}
El lenguaje de programación Rust es un lenguaje de programación innovador que prioriza la seguridad y la eficiencia. Este lenguaje ha sido diseñado por Mozilla Research, y se ha convertido en uno de los lenguajes de programación más populares en la actualidad. Rust es un lenguaje de programación multiparadigma, que combina elementos de programación funcional, orientada a objetos e imperativa. Este es un lenguaje de programación de tipado estático, que permite la inferencia de tipos, y que garantiza la seguridad de la memoria en tiempo de compilación.

Rust dispone de características modernas, como un gestor de paquetes para la gestión de dependencias, llamado Cargo, y compilación nativa a WebAssembly, lo que permite la ejecución de código Rust en un navegador web \autocite{WebAssembly}. Por tanto, cumple los objetivos del proyecto, al permitir la compilación a los sistemas operativos mayoritarios, y a un entorno web.

\subsubsection{WebAssembly y wasm-bindgen}

WebAssembly (WASM) es un formato binario de bajo nivel diseñado como destino de compilación para lenguajes de alto nivel, permitiendo ejecutar código con un rendimiento cercano al nativo en navegadores web \autocite{WebAssembly}. Rust es uno de los lenguajes con mejor soporte para WASM, gracias a su sistema de tipos y gestión de memoria sin recolector de basura.

La herramienta \texttt{wasm-bindgen} \autocite{IntroductionWasmbindgenGuide} facilita la interoperabilidad entre Rust y JavaScript, permitiendo el intercambio de datos complejos entre ambos lenguajes. Esta biblioteca automatiza gran parte del trabajo necesario para la comunicación entre el código Rust compilado a WASM y el entorno JavaScript del navegador, resolviendo una de las limitaciones mencionadas anteriormente: la falta de acceso directo al DOM.

Con \texttt{wasm-bindgen}, es posible exponer funciones Rust a JavaScript y viceversa, trabajar con tipos de JavaScript desde Rust, y manipular el DOM a través de interfaces que resultan naturales para los programadores de Rust. Esto hace que el desarrollo de aplicaciones web con Rust sea una alternativa viable y eficiente a JavaScript puro, especialmente para componentes que requieren alto rendimiento o seguridad de memoria.

\subsection{Otras herramientas consideradas}
\subsubsection{Unity}
Unity es un motor de videojuegos multiplataforma, que permite el desarrollo de videojuegos en 2D y 3D. Este motor es muy popular, que se puede utilizar también como herramienta para crear simulaciones, que es el propósito de este proyecto. Unity dispone de una gran cantidad de herramientas y bibliotecas, que facilitan el desarrollo de aplicaciones de todo tipo. Sin embargo, Unity es un motor de videojuegos pesado, que consume muchos recursos, y que no es especialmente eficiente en términos de rendimiento, sobre todo en entornos web.

\subsection{Comparativa de rendimiento}

Como se puede observar en la \autoref{fig:comparativa-rendimiento}, que coincide con el análisis de \autocite{samTop10Fastest2024}, Rust cae en los primeros puestos por rendimiento, teniendo un rendimiento similar al de C y C++, y superando ampliamente a Python y Node.js (\textit{JavaScript}).

\begin{figure}[H]
    \centering
    \includegraphics[width=\textwidth]{comparativa rendimiento lenguajes.png}
    \caption{Comparativa de rendimiento en entornos de escritorio \autocite{zotero-253}}
    \label{fig:comparativa-rendimiento}
\end{figure}

% https://goodmanwen.github.io/Programming-Language-Benchmarks-Visualization/

Sin embargo, esta tabla no es válida en entornos web, dado que en estos entornos el rendimiento de los diferentes lenguajes de programación se ve limitado por características propias de los navegadores. Por ejemplo, Rust compilado a WebAssembly no tiene acceso directo al \ac{DOM}, que es el modelo de objetos de un documento HTML. Esto significa que Rust no puede manipular directamente los elementos de la página web, y que necesita comunicarse con JavaScript para hacerlo. 

Un \textit{benchmark} de rendimiento en entornos web se puede encontrar en \autocite{InteractiveResults}, que muestra que \textit{vanilla JavaScript} es el lenguaje más rápido en este tipo de entornos. Sin embargo, en un proyecto como este, no se contempla el uso de \textit{vanilla JavaScript}, puesto que se necesita un \textit{framework} que permita la creación de aplicaciones gráficas de forma sencilla y eficiente. Si se analiza el rendimiento de diferentes \textit{frameworks} en entornos web, se puede ver que la diferencia entre los basados en \textit{JavaScript} y los basados en \textit{Rust} es mínima, no habiendo un claro vencedor en este aspecto \autocite{InteractiveResults}.

\subsection{Elección del lenguaje de programación}

Valorando las diferentes opciones, se ha decidido utilizar Rust para el desarrollo de la aplicación. Las características que han sido determinantes en la elección de Rust son las siguientes:

\begin{enumerate}
    \item \textbf{Seguridad:} Rust es un lenguaje de programación que garantiza la seguridad de la memoria en tiempo de compilación, lo que evita errores comunes en la programación, como los desbordamientos de búfer o las fugas de memoria.
    \item \textbf{Eficiencia:} Rust es un lenguaje de programación muy eficiente, que permite la programación de bajo nivel, y que garantiza un alto rendimiento en la ejecución del código.
    \item \textbf{Modernidad:} Rust es un lenguaje de programación moderno, que dispone de características avanzadas, como un gestor de paquetes para la gestión de dependencias, y compilación nativa a WebAssembly.
    \item \textbf{Versatilidad:} Rust es un lenguaje de programación multiparadigma, que combina elementos de programación funcional, orientada a objetos e imperativa, lo que facilita el desarrollo de aplicaciones de todo tipo.
\end{enumerate}

\section{Ficheros de configuración} \label{sec:ficheros-configuracion}

Los ficheros \ac{C3D} que este proyecto pretende representar pueden tener un número muy elevado de marcadores, superando ampliamente el millar. Representar una cantidad muy elevada de marcadores en el entorno es contraproducente, pues no permite focalizar en los puntos que se pretende estudiar. Por tanto, es importante generar una configuración que permita seleccionar los marcadores de interés.

Vicon cuenta con un formato propietario para definir configuraciones que su aplicación de representación de movimiento utiliza. En el \autoref{lst:mkr} se muestra un fragmento de este tipo de ficheros, (formato MKR), que presenta una única configuración de marcadores, llamada \texttt{Autolabel}.

\begin{lstlisting}[style=mystyle, caption={Fragmento de un fichero MKR}, label=lst:mkr]
!MKR#2

[Autolabel]
	LFHD	Left front head
	RFHD	Right front head
	LBHD	Left back head
	RBHD	Right back head

    LFHD,RFHD,LBHD,RBHD
\end{lstlisting}

En esta configuración, se definen cuatro marcadores, \texttt{LFHD}, \texttt{RFHD}, \texttt{LBHD} y \texttt{RBHD}, y se les añade una breve descripción. Estos marcadores serán los que la aplicación represente. En la última línea se especifica que estos marcadores se unirán con una línea, por estar separados mediante comas. Este formato es sencillo y fácil de entender, pero no es demasiado flexible, y no permite variar el tamaño, forma o color de los marcadores o uniones.

Para este proyecto, se ha decidido usar un formato estándar, con un mayor número de opciones que el formato MKR. Para este caso de uso existen diversos lenguajes de configuración comúnmente utilizados.

\subsection{\acs{JSON}}

\ac{JSON} es un formato de serialización de datos ligero y de fácil lectura para humanos y máquinas. Originalmente derivado de JavaScript, \ac{JSON} es un fichero comúnmente utilizado en numerosos entornos por su simplicidad.

Este formato utiliza una estructura basada en pares clave-valor y soporta tipos de datos básicos como números, cadenas de texto, booleanos, listas y objetos anidados. Su sintaxis compacta y la amplia disponibilidad de bibliotecas para su procesamiento en prácticamente todos los lenguajes de programación lo han convertido en una opción popular para la configuración de aplicaciones e intercambio de datos \autocite{ECMA404}.

A pesar de su popularidad, \ac{JSON} presenta algunas limitaciones como la falta de soporte nativo para comentarios y la necesidad de utilizar comillas para las claves, lo que puede reducir su legibilidad en configuraciones complejas.

\subsection{\acs{YAML}}

Igual que \ac{JSON}, \ac{YAML} es un formato de serialización de datos que se utiliza comúnmente para la configuración y el intercambio de datos. Sin embargo, \ac{YAML} se diseñó con un enfoque en la legibilidad humana, utilizando sangría para definir la estructura jerárquica en lugar de llaves y corchetes.

Este formato destaca por su sintaxis minimalista que elimina la mayoría de los símbolos de puntuación y permite representar datos complejos de manera clara. \ac{YAML} soporta todos los tipos de datos básicos de \ac{JSON}, pero añade características adicionales como soporte nativo para comentarios, referencias y etiquetas de tipo explícitas \autocite{YAMLAintMarkup}.

Una de las ventajas principales de \ac{YAML} es su capacidad para representar configuraciones complejas de manera concisa y legible. Es ampliamente utilizado en herramientas de infraestructura como código, configuración de aplicaciones y plataformas de contenedores como Docker y Kubernetes \autocite{FicheroDockercomposeyml,february2024YAMLBasicsKubernetes}.

Sin embargo, \ac{YAML} también presenta algunos inconvenientes, como la dependencia en la sangría que puede generar errores difíciles de detectar, y una sintaxis que en ocasiones puede resultar ambigua, especialmente en el manejo de cadenas de texto con caracteres especiales.

\subsection{\acs{TOML}}

\ac{TOML} es un formato de serialización de datos diseñado para ser un lenguaje de configuración sencillo y legible. \ac{TOML} es famoso por su uso en Rust, siendo el formato oficial para los ficheros de configuración \texttt{Cargo.toml} del gestor de paquetes de Rust.

Este formato se caracteriza por su sintaxis clara que recuerda a los ficheros INI pero con tipos de datos más ricos y soporte para estructuras jerárquicas. \ac{TOML} destaca por su capacidad para representar datos complejos de manera clara y sin ambigüedades, lo que lo hace ideal para configuraciones de aplicaciones \autocite{TOMLEnglishV100}.

\subsection{Elección del formato de configuración}

En este proyecto se ha decidido utilizar el formato \ac{TOML} para definir la configuración de representación. Este formato es sencillo y comparte similitudes con el formato MKR de Vicon. Además, \ac{TOML} es un formato de configuración ampliamente utilizado en el ecosistema de Rust, lo que da visibilidad a este lenguaje.

\section{Herramientas de desarrollo}

Para conseguir una aplicación gráfica eficiente, se debe de contar con una interfaz que permita a la aplicación trabajar con el hardware del dispositivo. Por tanto, es importante utilizar herramientas multiplataforma que permitan a la aplicación ser eficiente en diferentes entornos. En Rust, existen muchos \textit{crates} (unidades de compilación, ver \autocite{PackagesCratesRust}) que permiten el desarrollo de aplicaciones gráficas, como \texttt{wgpu} \autocite{WgpuPortableGraphics}, \texttt{glium} \autocite{GliumGlium2025}, \texttt{piston} \autocite{PistonModularOpen}, etc. Sin embargo, estos \textit{crates} son de muy bajo nivel, y crear una aplicación gráfica desde cero con ellos puede ser una tarea exigente en términos de tiempo. 

\subsection{Bevy} \label{subsec:bevy}
Para eliminar la limitación anteriormente comentada, se ha decidido utilizar un motor de videojuegos modular, que implemente internamente diferentes APIs gráficas, como Vulkan, Metal o DirectX, abstrayendo esta complejidad. Para este proyecto, se ha decidido utilizar el motor de videojuegos \textit{Bevy}, que está escrito plenamente en Rust y que cuenta con una comunidad numerosa y activa \autocite{BevyEngine}.

Las principales características de \texttt{Bevy} son las siguientes:
\begin{enumerate}
    \item \textbf{Eficiencia:} \texttt{Bevy} es un motor de videojuegos muy eficiente, que garantiza un alto rendimiento en la ejecución del código. Consigue esta eficiencia gracias a su diseño modular, que implica que sólo se compilarán los módulos del motor que el proyecto necesite. Utiliza programación concurrente siempre que es posible (esta es una limitación importante de los entornos web), y dispone de un sistema de eventos que permite la comunicación entre diferentes sistemas.
    \item \textbf{Soporte multiplataforma:} \texttt{Bevy} es un motor de videojuegos multiplataforma, que interacciona con diferentes APIs gráficas, como Vulkan, Metal o DirectX en entornos de escritorio, o WebGPU en entornos web. Esto permite que una aplicación desarrollada con \texttt{Bevy} pueda ejecutarse en diferentes sistemas operativos, como Windows, macOS, Linux o la web. La elección de la API gráfica se realiza de forma automática, en función del sistema operativo en el que se ejecute la aplicación.
\end{enumerate}

Otro aspecto destacado de \texttt{Bevy} es que se basa en el patrón de diseño ECS (\textit{Entity-Component-System}), que permite la creación de entidades, componentes y sistemas de forma independiente. 

La versión que este proyecto utiliza de \texttt{Bevy} es la 0.15.0, que fue la versión más reciente al momento de la implementación del proyecto.

\subsubsection{Paradigma \acs{ECS}} \label{subsec:ecs}

El paradigma \ac{ECS} es un patrón arquitectónico de software que ha ganado popularidad en el desarrollo de videojuegos y simulaciones por su eficiencia y flexibilidad \autocite{prdevingDeepdivingEntityComponent2023}. Diferentes artículos, como \autocite{zaksWhenNotUse2018} mencionan que este patrón es especialmente útil para aplicaciones interactivas.

A diferencia de los patrones tradicionales orientados a objetos, \ac{ECS} descompone la lógica de la aplicación en tres elementos principales \autocite{SistemaComponentesEntidad2024}:

\begin{itemize}
    \item \textbf{Entidades:} Son identificadores únicos que representan objetos en el mundo virtual. En sí mismas, las entidades no contienen datos ni comportamiento, funcionando más bien como contenedores lógicos para componentes.
    
    \item \textbf{Componentes:} Son estructuras de datos puras que contienen solo propiedades, sin comportamiento asociado. Cada componente representa un aspecto específico de una entidad (posición, renderizado, física, etc.). Una entidad puede tener múltiples componentes, siguiendo el principio de composición sobre herencia.
    
    \item \textbf{Sistemas:} Contienen la lógica que procesa entidades con combinaciones específicas de componentes. Cada sistema opera sobre un conjunto de componentes de forma independiente, realizando transformaciones en los datos.
\end{itemize}

En el contexto de Bevy, el paradigma \ac{ECS} se implementa de manera eficiente mediante un concepto adicional: los \textit{recursos}. Estos son datos globales accesibles por los sistemas sin estar asociados a entidades específicas. \textit{Bevy} también implementa un sistema de eventos que permite la comunicación entre sistemas sin acoplarlos directamente.

\subsection{EGUI} \label{subsec:egui}

Las aplicaciones de escritorio suelen implementar una \ac{GUI} que permite al usuario interactuar con la aplicación. Los \textit{frameworks} más utilizados para la creación de una \ac{GUI} son Qt \autocite{Theqtstory}, que se ha usado para crear aplicaciones famosas como el popular reproductor de vídeo y música VLC \autocite{QtGTKVideoLAN}, o GTK, que es el \textit{framework} utilizado por el entorno de escritorio GNOME, o el editor de fotografías GIMP. 

Pese a que estos \textit{frameworks} están escritos en C/C++, tienen \textit{bindings} a otros lenguajes de programación, como Rust. Sin embargo, para este proyecto se ha decidido utilizar un motor de videojuegos para facilitar el cálculo de la física y la representación gráfica. Por tanto, se ha decidido utilizar un \textit{framework} de \ac{GUI} que esté escrito en Rust y que sea compatible con \texttt{Bevy}. 

Un ejemplo de este tipo de \textit{frameworks} es \texttt{egui}, que es una biblioteca gráfica escrita en Rust, y que dispone compilación nativa a WebAssembly. Así, la misma \ac{GUI} es válida tanto para la aplicación de escritorio como para la versión web \autocite{ernerfeldtEmilkEgui2025}.

Adicionalmente, existe una adaptación de \texttt{egui} para \texttt{Bevy}, que permite la integración de la \ac{GUI} con el motor de videojuegos. Esta adaptación se llama \texttt{bevy\_egui} \autocite{Bevy_eguiRust}. 

\subsection{Arquitectura orientada a eventos} \label{subsec:eventos}

La arquitectura orientada a eventos es un patrón de diseño que organiza el flujo de un sistema en torno a la producción, detección y consumo de eventos, en lugar de seguir una secuencia lineal de instrucciones. Un evento representa un cambio en el estado del sistema o una señal externa que requiere atención \autocite{SistemaComponentesEntidad2024}.

En este patrón arquitectónico se identifican tres componentes principales:
\begin{itemize}
    \item \textbf{Emisores de eventos:} Entidades que generan y publican eventos cuando ocurre un cambio relevante en su estado o cuando detectan alguna condición específica.
    
    \item \textbf{Canales de eventos:} Infraestructura que permite transportar los eventos desde los emisores hasta los consumidores, pudiendo incluir colas, buses de eventos o mecanismos de publicación/suscripción.
    
    \item \textbf{Consumidores de eventos:} Componentes que reciben eventos y ejecutan lógica específica en respuesta a ellos, pudiendo a su vez generar nuevos eventos.
\end{itemize}

Esta arquitectura ofrece numerosas ventajas para aplicaciones interactivas, como:

\begin{itemize}
    \item \textbf{Acoplamiento débil:} Los emisores y consumidores de eventos no necesitan conocerse entre sí, lo que facilita el desarrollo y el mantenimiento de componentes individuales.
    
    \item \textbf{Escalabilidad:} Cada sistema puede reaccionar a múltiples eventos, siendo especialmente útil para aplicaciones con interfaces gráficas complejas o procesamiento asíncrono.
    
    \item \textbf{Reactividad:} Permite responder rápidamente a cambios en el estado de la aplicación o a acciones del usuario, relevante en los sistemas interactivos.
    
    \item \textbf{Extensibilidad:} Nuevas funcionalidades pueden añadirse creando nuevos consumidores de eventos existentes, sin modificar los componentes existentes.
\end{itemize}

\texttt{Bevy} implementa un sistema de eventos robusto que se integra perfectamente con su arquitectura \ac{ECS}. Los eventos en \texttt{Bevy} son estructuras de datos tipados que pueden ser enviados a través de un sistema de colas gestionado por el motor. Los sistemas pueden registrarse como receptores de tipos específicos de eventos y procesar solo aquellos relevantes para su funcionamiento \autocite{EventsBevyPrelude,EventsUnofficialBevy}. Esto permite una clara separación entre subsistemas mientras se mantiene un flujo de datos coherente.

\subsection{Fundamentos de representación 3D}

La representación tridimensional se basa en varios conceptos fundamentales. Las primitivas geométricas constituyen los elementos básicos utilizados para construir otros objetos complejos. Estos elementos se sitúan en sistemas de coordenadas cartesianas donde, por convenio, suele establecerse el eje Y vertical (contra la gravedad), aunque estas convenciones pueden variar entre diferentes aplicaciones.

La manipulación espacial de estos elementos requiere transformaciones geométricas (traslación, rotación y escalado), esenciales para posicionar correctamente los elementos a lo largo del tiempo. La percepción visual de estos elementos se consigue mediante técnicas de iluminación y asignación de materiales, que determinan características como el color o brillo. La selección adecuada de estos atributos mejora la percepción de profundidad y facilita la distinción entre diferentes elementos.


% Idea general

% \begin{itemize}
%     \item Parser C3D
%     \item Entorno Bevy
%     \item creación configuraciones (toml)
% \end{itemize}


\chapter{Desarrollo} \label{sec:cap3}

\noindent Este proyecto se ha llevado a cabo en diferentes etapas. En un comienzo, se trabaja con un fichero \ac{C3D} capturado por las cámaras de INEF, con el sistema de captura de datos de la marca Vicon explicado en el \autoref{sec:cap2}. Estos datos de entrada consisten en una grabación a 240 Hz de un \textit{swing} de golf\footnote{Un swing de golf es la acción mediante la cual los jugadores golpean la pelota en este deporte \autocite{GolfSwing2025}}. 

Como se ha explicado en el \autoref{sec:ficheros-c3d}, un \ac{C3D} es un fichero binario que contiene la posición de diferentes marcadores en el tiempo. Por tanto, no es difícil convertir estos datos en un formato de texto legible. Se muestra en la \autoref{tab:c3d_data} un fragmento de un fichero \ac{C3D} que se ha utilizado durante el desarrollo de la aplicación convertido a una tabla:

\begin{table}[htbp]
  \centering
  \setlength{\tabcolsep}{5pt}
  \renewcommand{\arraystretch}{1.2}
  \rowcolors{3}{white}{gray!10}
  \begin{tabular}{r|S[table-format=-3.2]S[table-format=-3.2]S[table-format=-3.2]|S[table-format=-3.2]S[table-format=-3.2]S[table-format=-3.2]}
  \toprule
  \multirow{2}{*}{\textbf{Frame}} & \multicolumn{3}{c|}{RFIN                          } & \multicolumn{3}{c}{RFRA                          } \\
    & {\textbf{x}} & {\textbf{y}} & {\textbf{z}} & {\textbf{x}} & {\textbf{y}} & {\textbf{z}} \\
  \midrule
  1 & 46.80 & 3.25 & 664.41 & -23.14 & -4.06 & 827.33 \\
  2 & 46.80 & 3.25 & 664.41 & -23.14 & -4.07 & 827.30 \\
  3 & 46.80 & 3.25 & 664.41 & -23.14 & -4.08 & 827.30 \\
  4 & 46.80 & 3.25 & 664.41 & -23.14 & -4.07 & 827.30 \\
  5 & 46.80 & 3.25 & 664.41 & -23.15 & -4.06 & 827.31 \\
  \midrule
  6 & 46.80 & 3.25 & 664.41 & -23.16 & -4.03 & 827.33 \\
  7 & 46.81 & 3.25 & 664.41 & -23.18 & -4.00 & 827.35 \\
  8 & 46.82 & 3.28 & 664.40 & -23.21 & -3.99 & 827.36 \\
  9 & 46.87 & 3.36 & 664.40 & -23.23 & -4.02 & 827.33 \\
  10 & 46.93 & 3.51 & 664.38 & -23.23 & -4.09 & 827.26 \\
  \bottomrule
  \end{tabular}
  \caption{Fragmento de un fichero C3D convertido a tabla}
  \label{tab:c3d_data}
\end{table}

Se puede observar que aparecen dos marcadores, \ac{RFIN} y \ac{RFRA}, con sus coordenadas en el espacio, representadas por los valores de las columnas \textit{x}, \textit{y} y \textit{z}, en unidades de milímetros.

\section{\textit{Parser} \acs{C3D}} \label{sec:parser-c3d}
Estos datos se deben integrar con el resto del programa. Para ello, se debe contar con un \textit{parser} que convierta trate con el binario.

Se ha partido del \textit{crate} de \textit{Rust} \texttt{bevy-c3d}. Este \textit{crate} utiliza a su vez el \textit{crate} \texttt{c3dio}, que es un \textit{parser} para ficheros \ac{C3D}, pero integrado a \textit{Bevy}. De este modo, se consigue un \textit{Asset} que se puede cargar en el motor de juego.

Se ha modificado el \textit{crate} \texttt{c3dio} para adaptarlo a las necesidades del proyecto. En este caso, se han arreglado dos errores graves que impiden la compatibilidad con los ficheros generado por Vicon. Ambos errores se deben a modificaciones en la especificación introducidos de manera unilateral por Vicon. El primero de ellos se debe a una modificación en la especificación que cambia la manera de codificar los eventos, que hace que no se lea ningún evento del \ac{C3D}. El segundo error se debe a que Vicon elimina una restricción de la especificación, que limita el número de marcadores a 255. En este caso, Vicon permite un número ilimitado de marcadores, lo que provoca que el \textit{parser} solo reconozca los primeros 255 marcadores, y el resto no se lean.   

\section{\textit{Bevy}} \label{sec:bevy}
Este proyecto usa el motor de videojuegos \textit{Bevy} como base para generar un entorno tridimensional donde representar el \ac{C3D}. Bevy sigue el paradigma \ac{ECS}. \todo{Referencia}

Se muestra en la \autoref{fig:entorno3D} una imagen del entorno tridimensional creado con Bevy, antes de cargar el \ac{C3D}.


\begin{figure}[H]
  \centering
  \includegraphics[width=\textwidth]{imagenes/entorno3D.png}
  \caption{Entorno 3D creado con Bevy (modo oscuro)}
  \label{fig:entorno3D}
\end{figure}

En este entorno destacan el plano central, representado en un color verde oscuro, y sobre el que se apoyan 3 vectores que sirven como referencia. El código que genera estos vectores se incluye en el \autoref{apx:spawn_ref_vec}, demostrando la flexibilidad de \textit{Bevy} y el potencial de Rust. 

El pequeño botón que aparece en la esquina superior izquierda permite cambiar el color de fondo, alternando entre un modo claro y un modo oscuro. En la \autoref{fig:modo_claro} se muestra el entorno en modo claro. Variar entre estos modos permite mejorar el contraste con los marcadores, y por tanto, mejorar la visibilidad de los mismos.

\begin{figure}[H]
  \centering
  \includegraphics[width=\textwidth]{imagenes/modo_claro.png}
  \caption{Entorno en modo claro}
  \label{fig:modo_claro}
\end{figure}

\subsection{Información global} \label{sec:bevy-global}

El programa almacena una serie de información a partir del \ac{C3D} cargado. Esta información se almacena en un \texttt{Resource}, que es una estructura de datos que permite gestionar y acceder a la información de manera eficiente. Esta estructura tiene el nombre de \texttt{AppState}, y contiene la información global necesaria para el funcionamiento del programa. Esta información incluye el \textit{frame} actual, el número de \textit{frames} del \ac{C3D}, el \textit{path} al fichero \ac{C3D}, el \textit{path} al fichero de configuración, la tasa de refresco del \ac{C3D}, la tasa de refresco seleccionada por el usuario, y diferentes booleanos que contienen información sobre el estado del programa, como si se está reproduciendo o no.


\subsection{Orientación a eventos} \label{sec:bevy-eventos}

Este proyecto tiene un enfoque orientado a eventos. Esto significa que el programa reacciona a los eventos que ocurren en el entorno, como la carga de un fichero o la interacción del usuario. Un evento en \textit{Bevy} es una estructura de datos que representa una acción o un cambio en el estado del programa. Estos eventos pueden ser generados por el usuario, como pulsar un botón o mover el ratón, o por el propio programa, como la carga de un fichero o la actualización de la escena. Cuando ocurre un evento, \textit{Bevy} reacciona en la siguiente iteración de su bucle interno, lo que implica que tiene efecto en la siguiente actualización de la imagen, que típicamente coincidirá con el siguiente \textit{frame}.

Los eventos definidos en este proyecto se agrupan en seis enumerados, que se reúnen en la \autoref{tab:enum-eventos}:

\begin{table}[H]
  \centering
  % \setlength{\tabcolsep}{5pt}
  \rowcolors{2}{white}{gray!10}
  \begin{tabular}{c|c}
  \toprule
  {\textbf{Grupo}} & {\textbf{Relación}} \\
  \midrule
  MarkerEvent & Marcadores \\
  JoinsEvent & Uniones \\
  VectorEvent & Vectores \\
  TraceEvent & Trazas \\
  MilestoneEvent & Eventos \\
  GraphEvent & Interfaz gráfica \\
  \bottomrule
  \end{tabular}
  \caption{Grupos de eventos definidos en el programa}
  \label{tab:enum-eventos}
\end{table}

Cada grupo de eventos cuenta con un orquestador encargado de tomar las acciones pertinentes dependiendo del subtipo de evento. Rust facilita el manejo de eventos de este tipo, puesto que cada grupo se codifica como un \texttt{enum}, que define diferentes variantes.

En resumen, los eventos definidos en el programa se agrupan en la \autoref{tab:eventos}:

\begin{table}[H]
  \centering
  % \setlength{\tabcolsep}{5pt}
  \rowcolors{2}{white}{gray!10}
  \begin{tabular}{c|c}
  \toprule
  {\textbf{Evento}} & {\textbf{Significado}} \\
  \midrule
  DespawnAllMarkersEvent & Elimina todos los marcadores \\
  DespawnAllJoinsEvent & Elimina todas uniones representadas \\
  DespawnJoinEvent(String, String) & Elimina la unión entre dos marcadores \\
  HideAllVectorsEvent & Esconder todos los vectores \\
  ShowAllVectorsEvent & Mostrar todos los vectores \\
  HideVectorEvent(Vector) & Esconder un vector \\
  ShowVectorEvent(Vector) & Mostrar un vector \\  
  AddTraceEvent(String) & Añade la traza de un marcador \\
  UpdateTraceEvent & Recargar las traza \\
  DespawnTraceEvent(String) & Eliminar la traza de un marcador \\
  DespawnAllTracesEvent & Eliminar todas las trazas\\
  AddMilestoneFromC3dEvent(usize) & Añade un evento en un \textit{frame} \\
  RemoveMilestoneEvent(usize) & Elimina el evento de un \textit{frame} \\
  RemoveAllMilestonesEvent & Elimina todos los eventos \\
  AddGraph(String, XYZ) & Crea un gráfico \\
  RemoveGraph(String) & Elimina un gráfico \\
  RestartGraphs & Reinicia los gráficos \\
  CreateMarkersWindow & Crea la ventana de selección \\  
  \bottomrule
  \end{tabular}
  \caption{Eventos definidos en el programa}
  \label{tab:eventos}
\end{table}

\section{Panel inferior} \label{sec:representacion-eventos}

El panel inferior de la pantalla está mayormente dedicado a la representación temporal del \ac{C3D}. 

\begin{figure}[H]
  \centering
  \includegraphics[width=\textwidth]{imagenes/panel_inf.png}
  \caption{Panel inferior de la representación temporal del \ac{C3D}}
  \label{fig:panel_inferior}
\end{figure}


Se distinguen tres filas. En la fila superior, en la parte izquierda aparece el \textit{frame} actual, representado mediante un \textit{slider}, que se puede mover para variar el \textit{frame}, y un número, ubicado justo a su derecha. Adicionalmente, a la derecha aparece otro \textit{slider} que permite ajustar la velocidad de reproducción de los eventos.

La segunda fila está dedicada a los eventos del \ac{C3D} (no confundir con los eventos de \textit{Bevy}), que son un parámetro dentro del fichero \ac{C3D} que permite fijar momentos importantes en el tiempo. Los eventos se pueden utilizar para navegar a través de estos momentos clave gracias a los botones de la parte derecha, que de izquierda a derecha significan: retroceder al evento anterior, marcar el \textit{frame} actual como un evento, variar entre pausa y reproducción, avanzar al evento siguiente, eliminar eventos y eliminar los eventos marcados por el usuario, dejando solo los eventos originales del \ac{C3D}.

Para diferenciar los eventos de \textit{Bevy} y los eventos del \ac{C3D}, se ha renombrado a estos últimos como \textit{milestones} (\textit{hitos}).

La última fila contiene un \textit{slider} que permite variar el rango de las trazas, \todo{Añadir cosas de trazas} variar entre configuraciones y ver u ocultar vectores, que se explican a continuación.

\section{Fichero de configuración \acs{TOML}}

El fichero de configuración de este proyecto define el modo de representación del fichero \ac{C3D} en el entorno 3D. 

Al igual que en el fichero de configuración de Vicon, explicado en \autoref{sec:ficheros-configuracion}, se definen las posibles configuraciones entre corchetes. Cualquier configuración cuenta con un grupo de puntos visibles (\textit{visible\_points}), un grupo de uniones (\textit{joins}) y un grupo de vectores (\textit{vectors}).

Estos grupos son opcionales, y pueden aparecer en cualquier orden. En caso de no aparecer, el programa no los representará.

\subsection{Configuración básica}

Una posible configuración podría ser la siguiente:

\begin{lstlisting}[style=mystyle, caption={Ejemplo simple de un fichero de configuración}, label={lst:ajuste-simple}]
[config1]
  visible_points = [
    "LFHD", "RFHD", "RBHD", "LBHD"
  ]
  joins = [
    ["LFHD", "RFHD", "RBHD", "LBHD", "LFHD"]
  ]
\end{lstlisting}

En el \autoref{lst:ajuste-simple} se muestra una configuración sencilla, \texttt{config1}, que representa los marcadores \ac{LFHD}, \ac{RFHD}, \ac{RBHD} y \ac{LBHD} como puntos visibles. Después, se define una unión entre los cuatro marcadores. Para representar un cuadrado, se duplica el primer marcador, \ac{LFHD}, al final de la lista. De este modo, se cierra la unión entre los cuatro marcadores. Además, se puede apreciar que en el grupo de uniones aparece un corchete extra. Esto se debe a que el grupo se considera una lista de listas. Las listas hijas representan diferentes uniones, es decir, grupos independientes de marcadores que se unen entre sí.

Estos cuatro marcadores representan los puntos de la cabeza. Es una combinación muy común, que posiblemente se utilice en varias configuraciones. Para facilitar su uso, el fichero de configuración permite definir un grupo de marcadores como un alias. De este modo, se puede definir el grupo de marcadores de la cabeza como \texttt{head}, y usarlo en la configuración de la siguiente manera:

\begin{lstlisting}[style=mystyle, caption={Ejemplo de un grupo de puntos}, label={lst:grupos-puntos}]
[point_groups]
  head = ["LFHD", "RFHD", "RBHD", "LBHD", "LFHD"]
  shoulders = ["LSHO", "RSHO"]

[config1]
  visible_points = [
    ["head"]
  ]
  joins = [
    [["head"]]
  ]
[config2]
  visible_points = [
    ["head"], ["shoulders"]
  ]
  joins = [
    [["head"]], 
    [["shoulders"]]
  ]
\end{lstlisting}

En el \autoref{lst:grupos-puntos} se muestra un ejemplo de cómo definir grupos de puntos, mediante la palabra reservada \textit{point\_groups}. En este caso, se definen dos grupos, \texttt{head} y \texttt{shoulders}. La forma de indicar al programa que se está referenciando a un grupo de puntos en vez de a un marcador es envolviendo al nombre entre corchetes.

En este caso, en la configuración \texttt{config1}, se representa el grupo de la cabeza como un punto visible y como una unión\footnote{Nótese que al usar el grupo de puntos de la cabeza, se está incluyendo dos veces el punto \ac{LFHD}. Esto no es relevante, dado que internamente el programa trata este punto como uno solo, al usarse como clave de un \textit{Hashmap}, que no permite la duplicación de elementos.}. En la configuración \texttt{config2}, se representan ambos grupos como puntos visibles y como uniones independientes. 

\subsection{Personalización general: Color y tamaño}

El fichero de configuración permite personalizar la representación de los marcadores. Para ello, se definen una serie de palabras reservadas. Estas palabras son \texttt{point\_color}, \texttt{point\_size}, \texttt{join\_color} y \texttt{line\_thickness}. Estas palabras son opcionales, y permiten personalizar la representación de los marcadores y las uniones. Los colores se pueden definir en formato rgb, como una lista de tres números enteros entre 0 y 255, o como rgb con transparencia, como una lista de cuatro números enteros entre 0 y 255.

\begin{lstlisting}[style=mystyle, caption={Configuración personalizada}, label={lst:cfg-personalizada}]
[point_groups]
  head = ["LFHD", "RFHD", "RBHD", "LBHD", "LFHD"]
  shoulders = ["LSHO", "RSHO"]

[config1]
  visible_points = [
    ["head"]
  ]
  joins = [
    [["head"]]
  ]
  point_color = [255, 0, 0]
  join_color = [0, 0, 255]
  line_thickness = 2.0
  point_size = 1.5

[config2]
  visible_points = [
    ["head"], ["shoulders"]
  ]
  joins = [
    [["head"]], 
    [["shoulders"]]
  ]
  point_color = [0, 0, 255, 100]
  point_size = 2.0
\end{lstlisting}

En el \autoref{lst:cfg-personalizada} se muestra un ejemplo de la personalización de una configuración, sobreescribiendo los valores por defecto. En este caso, para la configuración \texttt{config1}, los marcadores se representan en rojo, y las uniones en verde. Cada línea se representa con el doble de grosor que el valor por defecto, y los marcadores tienen un tamaño 1.5 veces mayor que el valor por defecto. En la configuración \texttt{config2}, los marcadores se representan en azul, con un 40\% de transparencia, y el doble de tamaño que el valor por defecto. Las uniones no se personalizan, por lo que se usan los valores por defecto.

Lo visto en el \autoref{lst:cfg-personalizada} es un ejemplo de sobreescritura de los valores por defecto para cada configuración. Sin embargo, se puede lograr una personalización más extrema gracias a los grupos de puntos, a los que se les pueden aplicar las mismas palabras reservadas. De este modo, se pueden definir diferentes colores y tamaños para cada grupo de puntos. En el \autoref{lst:cfg-grupos-personalizados} se muestra un ejemplo de cómo personalizar los grupos de puntos:

\begin{lstlisting}[style=mystyle, caption={Configuración personalizada de grupos de puntos}, label={lst:cfg-grupos-personalizados}]
[point_groups]
  head = ["LFHD", "RFHD", "RBHD", "LBHD", "LFHD"]
  shoulders = ["LSHO", "RSHO"]

[head.config]
  point_color = [255, 0, 0]
  point_size = 0.5
  join_color = [255, 128, 0]

[config1]
  visible_points = [
    ["head"]
  ]
  joins = [
    [["head"]]
  ]
  point_color = [255, 0, 0]
  join_color = [0, 0, 255]
  line_thickness = 2.0
  point_size = 1.5

[config2]
  visible_points = [
    ["head"], ["shoulders"]
  ]
  joins = [
    [["head"]], 
    [["shoulders"]]
  ]
  point_color = [0, 0, 255, 100]
  point_size = 2.0
\end{lstlisting}

Como se puede apreciar, se utiliza el nombre del grupo de puntos seguido de \texttt{.config} para definir la configuración de ese grupo, sobreescribiendo el valor por defecto de su configuración. En este caso, el grupo de puntos \texttt{head} se representa en rojo, con un tamaño de 0.5 veces el valor por defecto, y las uniones en naranja. El grupo de puntos \texttt{shoulders} no se personaliza, por lo que se usan los valores por defecto, que será azul en \texttt{config1} y verde en \texttt{config2}, puesto que verde es el color por defecto para las uniones, y en \texttt{config2} no se está sobreescribiendo.

\subsection{Personalización avanzada: Formas}

Las uniones entre dos marcadores se generan por defecto mediante un cilindro delgado, con apariencia de una línea. Mediante la palabra reservada \texttt{line\_thickness} se puede modificar este comportamiento, pero adicionalemente se puede modificar la forma. Las otras figuras admitidas son: un cono, un semicono y un prisma rectangular.

\begin{lstlisting}[style=mystyle, caption={Uso de formas}, label={lst:cfg-formas}] 
[config1]
  joins = [
    { points = ["RSJC", "RELJ"], shape = { type = "cone", radius = 1 } },
    { points = ["RELJ", "RWJC"], shape = { type = "semicone", radius1 = 4.0, radius2 = 2.0 } },
    { points = ["LSJC", "LELJ", "LWJC"], shape = { type = "prism", width = 4.0, height = 2.0 } },
  ]
\end{lstlisting}

Cada forma dispone de sus propias palabras reservadas, y las mayúsculas son irrelevantes (\textit{Case insensitive}). 

En el caso del cono, solo se debe especificar el radio, que define el tamaño de la circunferencia del primer elemento de la unión (en este caso \texttt{RSJC}). Además, cada forma tiene sinónimos, esto es, palabras que significan lo mismo para el programa. En el caso de ``cone'', se puede especificar también en español (``cono'').

En el caso del semicono, se deben especificar dos radios, que definen el tamaño de las dos circunferencias. Además, sinónimos de ``semicone'' son: ``semicono'', ``cone frustum'', ``cono truncado'', ``partial cone'', ``cono parcial'', ``truncated cone'' y ``cono truncado''.

\subsection{Configuración de vectores}

Los vectores son similares a los vectores de referencia explicados en \autoref{sec:bevy}. En el fichero de configuración, se definen como un punto de anclaje, una lista de elementos anclados a ese marcador, y opcionalmente una escala. Los vectores permiten representar relaciones espaciales y se pueden utilizar para definir trayectorias o movimientos en el espacio tridimensional.

\begin{lstlisting}[style=mystyle, caption={Configuración de un vector}, label={lst:cfg-vector}]
[config1]
  vectors = [ 
    ["OBJ1", "LVelOBJ1"],
    ["LUarmCM", ["LUarmIv", "LUarmJv", "LUarmKv"], 2.5],
    ["RUarmCM", ["RUarmIv", "RUarmJv"], 2.5],
    ["RUarmCM", "RUarmKv", 1.5],
]
\end{lstlisting}

En el \autoref{lst:cfg-vector} se muestra un ejemplo de como configurar vectores. En este caso, se han definido tres puntos de anclaje: \texttt{OBJ1}, \texttt{LUarmCM} y \texttt{RUarmCM}. 

El primero de ellos, \texttt{OBJ1}, tiene un solo elemento anclado, \texttt{LVelOBJ1}. Esta configuración ancla el marcador de representación de la velocidad de LVelOBJ1 a OBJ1. Esto representará un vector tangente a la trayectoria del objeto en cualquier punto.

El segundo, \texttt{LUarmCM}, tiene tres elementos anclados, \texttt{LUarmIv}, \texttt{LUarmJv} y \texttt{LUarmKv}, y una escala de 2.5. Esta es una configuración estándar para la representación de un sistema de coordenadas local, que sirven para cuantificar la rotación de un marcador.

El tercero, \texttt{RUarmCM}, tiene tres elementos anclados, \texttt{RUarmIv} y \texttt{RUarmJv}, con una escala de 2.5, y posteriormente se añade el tercer elemento, \texttt{RUarmKv}, con una escala de 1.5. Similar al caso anterior, se representa un sistema de coordenadas local, pero reduciendo el tamaño de la componente K.

Los vectores tienen una configuración de color fija. Para los vectores que tienen un número diferente a tres elementos anclados, se representan en amarillo, mientras que si el número de marcadores anclados a un punto es exactamente igual a tres, el programa entiende que se trata de una representación de vectores de referencia, para lo que utiliza colores estándar en otros motores de videojuegos, representando la primera componente en rojo, la segunda en verde y la tercera en azul. 

\subsection{Expresiones regulares}

A la hora de leer el fichero de configuración, se utilizan expresiones regulares para determinar si un marcador debe representarse o no. Esto permite representar un grupo de marcadores sin necesidad de definirlos uno a uno. Sin embargo, esto podría causar que se añadan más marcadores de los deseados. Por ejemplo, si se incluye el marcador \texttt{OBJ1}, se podría incluir también \texttt{LVelOBJ1}. 

Para evitar esto, el programa rodea el nombre del marcador con el símbolo \texttt{\^} al principio y el símbolo \texttt{\$} al final. De este modo, se asegura que el nombre del marcador coincide exactamente con el nombre del marcador en el fichero \ac{C3D}. Pero podría darse el caso de que se quiera representar ambos marcadores. Para ello, el programa admite un guión bajo (\texttt{\_}) como comodín. De este modo, si se incluye el marcador \texttt{\_OBJ1}, se incluirán todos los marcadores que cumplan la expresión regular, como \texttt{OBJ1} o \texttt{LVelOBJ1}. 

\section{Carga de ficheros}

El fichero de configuración y el fichero \ac{C3D} se cargan de forma simple, arrastrando el fichero a la ventana del programa. El programa detecta el tipo de fichero y lo carga automáticamente. Se puede variar el fichero \ac{C3D} o el fichero de configuración en tiempo de ejecución, y el programa lo detecta automáticamente.

\textit{Bevy} tiene una función nativa para la captura de los ficheros que se sueltan sobre la ventana. Sin embargo, en la versión web el navegador captura el fichero antes que \textit{Bevy}, por lo que es imposible la carga de ficheros de este modo. Para ello se ha modificado el crate \texttt{bevy\_web\_file\_drop} para darle compatibilidad con la versión 0.15 de \textit{Bevy}, dado que este \textit{crate} está escrito para la versión 0.14 y no presenta compatibilidad \autocite{Bevy_web_file_dropCratesioRust2024}.

Con estas modificaciones, el programa es capaz de capturar un archivo tanto en la versión de escritorio como en la versión web. Para esta última utiliza una \textit{Blob URL} para la carga del fichero \autocite{AnswerWhatBlob2015,FileAPI}.

Cuando un fichero se carga, se toman diferentes acciones, dependiendo de su tipo. Si se trata de un \ac{C3D} se eliminan todos los marcadores y se envía un evento de carga al \textit{parser} de \ac{C3D}. Este \textit{parser} se encarga de cargar el fichero y enviar un evento al programa con los datos cargados. Si se trata de un fichero de configuración, simplemente se indica al programa que debe generar nuevamente las uniones y vectores, sin necesidad de eliminar los marcadores. 

\section{Representación tridimensional} 

\subsection{Representación de los marcadores} \label{sec:representacion-marcadores}

Los marcadores se representan como esferas, y se pueden personalizar mediante el fichero de configuración. Internamente, el programa utiliza dos estructuras de datos, una contiene el nombre del marcador y su visibilidad, y la otra es una agrupación de marcadores, siguiendo el patrón \ac{ECS} de \textit{Bevy}. Es decir, el componente \texttt{Marker} contiene el nombre del marcador y su visibilidad, mientras que el componente \texttt{C3dMarkers} es únicamente una asignación de una estructura vacía a todos los marcadores. Esto permite eliminar todos los marcadores de forma sencilla, eliminando únicamente el componente padre.

Es necesario mantener en memoria todos los marcadores, incluso los que no son visibles según el fichero de configuración. Esto se debe a que el programa permite cambiar la configuración en tiempo de ejecución, y por tanto, es necesario mantener todos los marcadores en memoria para poder representarlos. Sin embargo, el programa no los representa si no son visibles, gracias a que la estructura de datos contiene la visibilidad de cada marcador\footnote{\textit{Bevy} entiende la visibilidad de cualquier entidad como un enumerado, que puede ser visible, invisible o heredado.}.

Si se ejecuta el programa sin un fichero de configuración, se representan todos los marcadores como visibles. Esto permite comprobar que el \textit{parser} de \ac{C3D} funciona correctamente, y que se están cargando todos los marcadores. Sin embargo, no se recomienda su uso, dado que el número de marcadores puede ser elevado, y la representación muestra datos que no tienen un sentido espacial. En la \autoref{fig:marcadores} se muestra una imagen de la representación de los marcadores sin un fichero de configuración. Se puede observar que el número de marcadores es elevado, y que aparecen marcadores que no tienen un sentido espacial, como un vector de velocidad. La \autoref{fig:marcadores} resalta la importancia de la configuración, puesto que es inviable el estudio de un movimiento con un número tan elevado de marcadores. 

\begin{figure}[H]
  \centering
  \includegraphics[width=\textwidth]{imagenes/marcadores.png}
  \caption{Representación de marcadores del \ac{C3D} sin un fichero de configuración}
  \label{fig:marcadores}
\end{figure}

Para representar los marcadores, el programa utiliza un componente de \textit{Bevy} llamado \texttt{Mesh}. Este componente permite representar una malla en el espacio tridimensional. En este caso, se utiliza una esfera como malla, con un tamaño y color definidos en el fichero de configuración, o en su defecto, en color azul y con un tamaño por defecto. Si se ha cargado un fichero de configuración, cada marcador se considera visible si aparece en el grupo de puntos visibles. En caso contrario, se considera invisible. De este modo, es sencillo representar solo una parte de los marcadores, y ocultar el resto.

\begin{figure}[H]
  \centering
  \includegraphics[width=\textwidth]{imagenes/config-basica.png}
  \caption{Representación de marcadores del \ac{C3D} con un fichero de configuración}
  \label{fig:cfg-basica}
\end{figure}

La \autoref{fig:cfg-basica} muestra el mismo fichero que la \autoref{fig:marcadores}, pero filtrando la mayoría de puntos para generar un avatar.

\subsection{Representación de las uniones} \label{sec:representacion-uniones}

Las uniones se representan como figuras entre marcadores. Estas figuras pueden ser de diferentes tipos, como un cilindro, un cono, un semicono o un prisma rectangular. 




\subsection{Velocidad de reproducción}

El programa admite dos modos de funcionamiento, velocidad fija o velocidad adaptativa. 

El primer modo es ideal para el estudio de un movimiento con la mayor precisión, esto es, representando la velocidad real de la captura del movimiento, utilizando la tasa de refresco del \ac{C3D}. En caso de que la tasa de refresco de la pantalla sea inferior a la tasa de refresco del \ac{C3D}, \textit{Bevy} realiza una interpolación para que la velocidad de reproducción sea la del \ac{C3D}. Por ejemplo, para una tasa de refresco de una pantalla de 60 Hz representando un movimiento de 240 Hz, \textit{Bevy} representará uno de cada cuatro fotogramas del \ac{C3D}. Además, este modo permite variar la velocidad de reproducción, entre 0.1 y 2.0 veces la tasa de refresco del \ac{C3D}. 

Por el contrario, el segundo modo es ideal para el estudio del movimiento con la mayor calidad, es decir, sin perder información sobre los fotogramas. En este modo, no se tiene en cuenta la tasa de refresco del \ac{C3D}, \textit{Bevy} representará el fichero respetando las limitaciones del \textit{hardware}. Esto es, si se dispone de una pantalla de 60 Hz, como máximo el programa se actualizará a esta velocidad (aunque si hay un componente más restrictivo, por ejemplo una tarjeta gráfica incapaz de dar una tasa de refresco adecuada, se respetará esta limitación).

% \section{Configuración de la cámara} \label{sec:cfg-camara}


% PRESUPUESTO (obligatorio)
\chapter{Presupuesto}
\label{sec:cap4}
\noindent Es \textbf{OBLIGATORIO}.
Se indicarán los costes de prototipos, trabajos, mano de obra,… necesarios para la realización del diseño. Igual que en el caso de los planos, se puede incorporar en el cuerpo del informe o como anexo. 


% Impacto del proyecto (obligatorio)
\chapter{Impacto del proyecto}
\label{sec:cap5}
\noindent Es OBLIGATORIO.
En este capítulo el estudiante pondrá de manifiesto las implicaciones sociales, de salud y seguridad, ambientales, económicas, tecnológicas o industriales que estén relacionadas son su trabajo, así como la posible aportación a los ODS (Objetivos de Desarrollo Sostenible.

Referencias útiles para elaborar este capítulo:
\begin{itemize}
    \item Naciones Unidas. Objetivos de Desarrollo Sostenible. Online: https://www.un.org/sustainabledevelopment/es/objetivos-de-desarrollo-sostenible/
    \item Fernández Aller, Celia; Miñano, Rafael. "Guía para trabajar la Responsabilidad Social y Ambiental (GRSA)". ETSI Sistemas Informáticos, UPM. (Versión 2.0 Mayo 2015). Online: 
    
    \href{https://oa.upm.es/35542/1/Guia_Responsabilidad_Social_y_Ambiental-V2-1.pdf}{https://oa.upm.es/35542/1/Guia\_Responsabilidad\_Social\_y\_Ambiental-V2-1.pdf}
    \begin{itemize}
        \item Páginas especialmente útiles para estudiantes de PFG: 31 a 40.
    \end{itemize}
\end{itemize}



% Conclusiones (obligatorio)
\chapter{Conclusiones y trabajo futuro}
\label{sec:cap6}
\noindent En el último apartado de la memoria se ha de incluir una visión general del trabajo realizado: problema propuesto, solución planteada y resultados obtenidos. Junto con esta descripción, también hay que especificar qué conocimiento nuevo se puede derivar de toda la información expuesta en el informe: validez o no del sistema, nuevos aspectos del problema detectados en el trabajo, etc. Finalmente, si el proyecto no deja totalmente cerrado y resuelto el problema tecnológico abordado, el proyectista debe aprovechar su experiencia para proponer líneas futuras de trabajo.

\section{Conclusiones}

\noindent Tras la realización de este proyecto fin de carrera ... 

\section{Trabajos futuros}

\noindent Bla bla bla...

\label{sec:bibliografía}
\printbibliography

% Anexo (obligatorio)
\appendix
\label{sec:apendice}
\chapter{Preguntas Frecuentes}
\chapter{Preguntas Frecuentes}

\noindent Se describen a continuación las preguntas frecuentes que pueden surgir al utilizar la aplicación. Estas preguntas se han recopilado de los comentarios y sugerencias de los usuarios durante el desarrollo del proyecto. Se espera que estas preguntas y respuestas sean útiles para los usuarios y ayuden a resolver dudas comunes.
\chapter{Herramientas utilizadas}
\input{capitulos/herramientas}
\chapter{Código fuente}
\noindent No se va a publicar el código fuente. Se adjuntan utilidades y algunas peculiaridades del proyecto.

\section{Conversor C3D a tabla} \label{apx:c3d_latex_py}

Esta utilidad sirve para convertir un fichero C3D a una tabla en \LaTeX, con un número personalizado de filas y columnas. Durante la redacción de la memoria, se ha usado para generar la \autoref{tab:c3d_data}.

\lstinputlisting[style=mystyle, language=Python, caption={Conversor C3D a tabla latex}]{utils/convert_c3d_to_latex.py}
\newpage
\section{Generación de vectores} \label{apx:spawn_ref_vec}

En este apéndice se presenta el código utilizado para la generación de vectores tridimensionales en el entorno visual. \textit{Bevy}, como motor de videojuegos, proporciona figuras geométricas primitivas como cilindros y conos, pero no ofrece una implementación directa para representar vectores direccionales, dado que su enfoque prioriza la ligereza antes que un paradigma totalmente funcional con todas las abstracciones preincorporadas.

La implementación mostrada a continuación combina estas primitivas geométricas (un cilindro para el cuerpo y un cono para la punta) para crear una representación visual de vectores. El código aprovecha características avanzadas de Rust como iteradores, closures y manejo de tipos de datos algebraicos, demostrando la expresividad y seguridad del lenguaje. Además, ilustra cómo extender las funcionalidades de \textit{Bevy} para adaptarlas a las necesidades específicas del proyecto.

\begin{lstlisting}[style=mystyle, language=Rust]
    fn spawn_reference_vectors(
        commands: &mut Commands, 
        materials: &mut ResMut<Assets<StandardMaterial>>, 
        meshes: &mut ResMut<Assets<Mesh>>
    ) {
        // Vectors for reference
        let default_cylinder_height = 0.25;
        let mut cone_mesh = Mesh::from(Cone {
            radius: 0.025,
            height: 0.1,
        });
        let mut vector_mesh = Mesh::from(Cylinder::new(
            0.01,
            default_cylinder_height,
        ));
    
        // Extract and modify positions
        if let Some(positions) = cone_mesh.attribute(Mesh::ATTRIBUTE_POSITION) {
            let modified_positions: Vec<[f32; 3]> = positions
                .as_float3()
                .unwrap_or(&[[0.0, 0.0, 0.0]])
                .iter()
                .map(|&[x, y, z]| [x, y + default_cylinder_height/2.0, z]) // cylinder height / 2, to place the cone on top of the cylinder (0 is the center of the cylinder)
                .collect();
            // Replace the positions attribute
            cone_mesh.insert_attribute(Mesh::ATTRIBUTE_POSITION, modified_positions);
        }
    
        vector_mesh.merge(&cone_mesh);
        
        let colors = vec![
            Color::srgb_u8(255, 0, 0),
            Color::srgb_u8(0, 255, 0),
            Color::srgb_u8(0, 0, 255),
        ];
        let point = Vec3::new(-2.0, -2.0, 0.0);
        let positions = vec![
            point + Vec3::new(default_cylinder_height / 2.0, 0.0, 0.0),
            point + Vec3::new(0.0, default_cylinder_height / 2.0, 0.0),
            point + Vec3::new(0.0, 0.0, default_cylinder_height / 2.0),
        ];
        let rotations = vec![
            Quat::from_axis_angle(Vec3::Z, -std::f32::consts::FRAC_PI_2),
            Quat::IDENTITY,
            Quat::from_axis_angle(Vec3::X, std::f32::consts::FRAC_PI_2),
        ];
        colors
            .iter()
            .zip(rotations)
            .zip(positions)
            .all(|((color, rotation), position)| {
            commands.spawn((
                Mesh3d(meshes.add(vector_mesh.clone())),
                MeshMaterial3d(materials.add(StandardMaterial {
                    base_color: color.clone(),
                    ..default()
                })),
                Transform::from_translation(position)
                    .with_rotation(rotation),
            ));
            true
        });
    }
\end{lstlisting}

\end{document}