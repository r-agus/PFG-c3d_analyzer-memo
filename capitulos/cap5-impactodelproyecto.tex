\chapter{Impacto del proyecto} \label{sec:cap5}

%En este capítulo el estudiante pondrá de manifiesto las implicaciones sociales, de salud y seguridad, ambientales, económicas, tecnológicas o industriales que estén relacionadas son su trabajo, así como la posible aportación a los ODS (Objetivos de Desarrollo Sostenible.

% Referencias útiles para elaborar este capítulo:
% \begin{itemize}
%     \item Naciones Unidas. Objetivos de Desarrollo Sostenible. Online: https://www.un.org/sustainabledevelopment/es/objetivos-de-desarrollo-sostenible/
%     \item Fernández Aller, Celia; Miñano, Rafael. "Guía para trabajar la Responsabilidad Social y Ambiental (GRSA)". ETSI Sistemas Informáticos, UPM. (Versión 2.0 Mayo 2015). Online: 
    
%     \href{https://oa.upm.es/35542/1/Guia_Responsabilidad_Social_y_Ambiental-V2-1.pdf}{https://oa.upm.es/35542/1/Guia\_Responsabilidad\_Social\_y\_Ambiental-V2-1.pdf}
%     \begin{itemize}
%         \item Páginas especialmente útiles para estudiantes de PFG: 31 a 40.
%     \end{itemize}
% \end{itemize}

\noindent En este capítulo se describen las implicaciones sociales, de salud y seguridad, ambientales, económicas, tecnológicas o industriales que están relacionadas con el trabajo realizado en este proyecto. Además, se analiza la posible aportación a los Objetivos de Desarrollo Sostenible (ODS) de la ONU.

\section{Contexto del proyecto}

Dentro del sector tecnológico, el proyecto se enmarca en el ámbito de la visión por ordenador y el análisis de datos. Por otro lado, en el ámbito organizativo y estratégico, el proyecto se sitúa en el contexto de la investigación y desarrollo de un producto. Respecto al ciclo de vida del proyecto, no se necesita una fase de extracción de recursos naturales, ya que el proyecto se basa en el desarrollo de software y no requiere la extracción de recursos físicos. Tampoco se necesita una fase de empaquetamiento o distribución, ya que el producto final es una aplicación de software que se puede distribuir a través de Internet. 

Sobre el contexto socio-económico, geográfico y cultural, el proyecto se desarrolla en un entorno académico y de investigación, donde se busca mejorar la comprensión del movimiento humano y su aplicación en el ámbito deportivo. 

Los grupos de interés relativos al proyecto son los investigadores, que se benefician de la tecnología de captura de movimiento para sus estudios, y los atletas y entrenadores, que pueden utilizar la aplicación para mejorar su rendimiento y prevenir lesiones. Además, el proyecto también puede beneficiar a la industria del deporte y la salud, al proporcionar herramientas avanzadas para el análisis del movimiento humano.

\section{Aspectos éticos}

El proyecto se desarrolla en un entorno académico y de investigación, donde se busca mejorar la comprensión del movimiento humano y su aplicación en el ámbito deportivo. Por lo tanto, es importante tener en cuenta los aspectos éticos relacionados con la investigación y el uso de tecnología de captura de movimiento.

La tecnología de captura de movimiento plantea cuestiones éticas relacionadas con la privacidad y la protección de datos personales. Es fundamental garantizar que los datos recopilados durante el proceso de captura de movimiento se manejen de manera responsable y se utilicen únicamente con fines de investigación y desarrollo. Además, es importante obtener el consentimiento informado de los participantes antes de realizar cualquier estudio que implique la captura de su movimiento.

Por último, es importante considerar el impacto de la tecnología de captura de movimiento en la sociedad en general. La tecnología de captura de movimiento puede tener un impacto significativo en la forma en que se entiende y se estudia el movimiento humano, lo que puede tener implicaciones para la salud y el bienestar de las personas. De este modo, es fundamental garantizar que la tecnología se utilice de manera responsable y ética.

\section{Implicaciones sociales}

El proyecto podría tener un impacto social significativo, ya que la tecnología de captura de movimiento se utiliza en una variedad de campos, incluyendo la medicina, el deporte y la educación. La aplicación se centra en el estudio del movimiento desde un enfoque de mejorar el rendimiento deportivo y la prevención de lesiones, lo que puede beneficiar a atletas y entrenadores al proporcionar información valiosa sobre la biomecánica del movimiento humano. Además, la tecnología de captura de movimiento también se utiliza en la rehabilitación médica, lo que puede ayudar a los pacientes a recuperar su movilidad y mejorar su calidad de vida.

De este modo, el proyecto contribuye a la promoción de un estilo de vida saludable y activo, en consonancia con el ODS 3, que busca garantizar una vida sana y promover el bienestar para todos en todas las edades. También se alinea con el ODS 4, que busca garantizar una educación inclusiva, equitativa y de calidad, y promover oportunidades de aprendizaje durante toda la vida para todos, ya que la tecnología de captura de movimiento puede ser utilizada en entornos educativos para enseñar sobre la biomecánica del movimiento humano. Por último, el proyecto también puede contribuir al ODS 9, que busca construir infraestructuras resilientes, promover la industrialización inclusiva y sostenible y fomentar la innovación, al desarrollar una aplicación que utiliza tecnología avanzada para el análisis del movimiento humano \autocite{ODSObjetivosDesarrollo}.

