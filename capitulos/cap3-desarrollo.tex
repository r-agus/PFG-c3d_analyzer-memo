% Idea general

% \begin{itemize}
%     \item Parser C3D
%     \item Entorno Bevy
%     \item creación configuraciones (toml)
% \end{itemize}


\chapter{Desarrollo} \label{sec:cap3}

\noindent Este proyecto se ha llevado a cabo en diferentes etapas. En un comienzo, se trabaja con un fichero \ac{C3D} capturado por las cámaras de INEF, con el sistema de captura de datos de la marca Vicon explicado en el \autoref{sec:cap2}. Estos datos de entrada consisten en una grabación a 240 Hz de un \textit{swing} de golf\footnote{Un swing de golf es la acción mediante la cual los jugadores golpean la pelota en este deporte \autocite{GolfSwing2025}}. 

Como se ha explicado en el \autoref{sec:ficheros-c3d}, un \ac{C3D} es un fichero binario que contiene la posición de diferentes marcadores en el tiempo. Por tanto, no es difícil convertir estos datos en un formato de texto legible. Se muestra en la \autoref{tab:c3d_data} un fragmento de un fichero \ac{C3D} que se ha utilizado durante el desarrollo de la aplicación convertido a una tabla:

\begin{table}[htbp]
  \centering
  \setlength{\tabcolsep}{5pt}
  \renewcommand{\arraystretch}{1.2}
  \rowcolors{3}{white}{gray!10}
  \begin{tabular}{r|S[table-format=-3.2]S[table-format=-3.2]S[table-format=-3.2]|S[table-format=-3.2]S[table-format=-3.2]S[table-format=-3.2]}
  \toprule
  \multirow{2}{*}{\textbf{Frame}} & \multicolumn{3}{c|}{RFIN                          } & \multicolumn{3}{c}{RFRA                          } \\
    & {\textbf{x}} & {\textbf{y}} & {\textbf{z}} & {\textbf{x}} & {\textbf{y}} & {\textbf{z}} \\
  \midrule
  1 & 46.80 & 3.25 & 664.41 & -23.14 & -4.06 & 827.33 \\
  2 & 46.80 & 3.25 & 664.41 & -23.14 & -4.07 & 827.30 \\
  3 & 46.80 & 3.25 & 664.41 & -23.14 & -4.08 & 827.30 \\
  4 & 46.80 & 3.25 & 664.41 & -23.14 & -4.07 & 827.30 \\
  5 & 46.80 & 3.25 & 664.41 & -23.15 & -4.06 & 827.31 \\
  \midrule
  6 & 46.80 & 3.25 & 664.41 & -23.16 & -4.03 & 827.33 \\
  7 & 46.81 & 3.25 & 664.41 & -23.18 & -4.00 & 827.35 \\
  8 & 46.82 & 3.28 & 664.40 & -23.21 & -3.99 & 827.36 \\
  9 & 46.87 & 3.36 & 664.40 & -23.23 & -4.02 & 827.33 \\
  10 & 46.93 & 3.51 & 664.38 & -23.23 & -4.09 & 827.26 \\
  \bottomrule
  \end{tabular}
  \caption{Fragmento de un fichero C3D convertido a tabla}
  \label{tab:c3d_data}
\end{table}

Se puede observar que aparecen dos marcadores, \ac{RFIN} y \ac{RFRA}, con sus coordenadas en el espacio, representadas por los valores de las columnas \textit{x}, \textit{y} y \textit{z}, en unidades de milímetros.

\section{\textit{Parser} \acs{C3D}} \label{sec:parser-c3d}
Estos datos se deben integrar con el resto del programa. Para ello, se debe contar con un \textit{parser} que convierta trate con el binario.

Se ha partido del \textit{crate} de \textit{Rust} \texttt{bevy-c3d}. Este \textit{crate} utiliza a su vez el \textit{crate} \texttt{c3dio}, que es un \textit{parser} para ficheros C3D, pero integrado a \textit{Bevy}. De este modo, se consigue un \textit{Asset} que se puede cargar en el motor de juego.

\section{\textit{Bevy}} \label{sec:bevy}
Este proyecto usa el motor de videojuegos \textit{Bevy} como base para generar un entorno tridimensional donde representar el \ac{C3D}. Bevy sigue el paradigma \ac{ECS} \todo{Referencia}

Se muestra en la \autoref{fig:entorno3D} una imagen del entorno tridimensional creado con Bevy, antes de cargar el \ac{C3D}.


\begin{figure}[H]
  \centering
  \includegraphics[width=\textwidth]{imagenes/entorno3D.png}
  \caption{Entorno 3D creado con Bevy}
  \label{fig:entorno3D}
\end{figure}

En este entorno destacan el plano central, representado en un color verde oscuro, y sobre el que se apoyan 3 vectores que sirven como referencia. El código que genera estos vectores se incluye en el \autoref{apx:spawn_ref_vec}, demostrando la flexibilidad de \textit{Bevy} y el potencial de Rust

\todo{Figura del swing en el entorno de Enrique}