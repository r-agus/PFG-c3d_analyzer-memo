\noindent En el último apartado de la memoria se ha de incluir una visión general del trabajo realizado: problema propuesto, solución planteada y resultados obtenidos. Junto con esta descripción, también hay que especificar qué conocimiento nuevo se puede derivar de toda la información expuesta en el informe: validez o no del sistema, nuevos aspectos del problema detectados en el trabajo, etc. Finalmente, si el proyecto no deja totalmente cerrado y resuelto el problema tecnológico abordado, el proyectista debe aprovechar su experiencia para proponer líneas futuras de trabajo.

\section{Conclusiones}

\noindent Tras la realización de este proyecto fin de carrera ... 

\section{Trabajos futuros}

\noindent Bla bla bla...