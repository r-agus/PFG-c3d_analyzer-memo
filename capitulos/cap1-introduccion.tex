\chapter{Introducción} \label{sec:cap1}

\section{Marco y motivación del proyecto}

% \noindent La introducción ha de hacer una resumida descripción del marco tecnológico donde se ubica el proyecto, justificando su necesidad y los objetivos que se buscan con su realización, es decir, del problema que se pretende resolver o analizar. También se puede incluir una visión general de las soluciones propuestas, sin entrar en detalles. La introducción debe concluir con una breve descripción (de no más de una página) de la estructura del informe del proyecto, resumiendo de manera escueta los contenidos de cada capítulo. Es frecuente que la introducción sea el último apartado del informe que se escriba.

\noindent El presente proyecto fin de grado se enmarca dentro del ámbito de las telecomunicaciones y el deporte, y tiene como objetivo el desarrollo de una aplicación para la visualización de datos en 3D para el estudio del movimiento.

El proyecto parte de una colaboración entre la facultad de \ac{INEF} de la \ac{UPM} y el centro de investigación \ac{GAMMA}, ubicado en el campus sur de la \ac{UPM}. Este centro tiene experiencia en el desarrollo de aplicaciones para la captura de movimiento con fines de investigación, usando tanto sistemas de captura con marcadores como sistemas de captura sin marcadores.

La colaboración entre ambas instituciones aprovecha la experiencia de \ac{INEF} en el ámbito deportivo y la experiencia del \ac{GAMMA} en el desarrollo de aplicaciones para la captura de movimiento.

\section{Objetivos técnicos y académicos}

Los objetivos de este proyecto fin de carrera son, desde el punto de vista técnico:

\begin{itemize}
    \item Desarrollar una aplicación para la visualización de datos biomecánicos en un entorno tridimensional.
    \item Implementar un sistema de síntesis de dichos datos, que permita el estudio de los mismos.
    \item Obtener estadísticas sobre los datos obtenidos, plasmadas sobre gráficos bidimensionales.
    \item Disponer de una aplicación multiplataforma, que se pueda compilar en los sistemas operativos más comunes (Windows, Linux y MacOS) y disponer de una versión web bajo el mismo código fuente.
    \item Fomentar el desarrollo de aplicaciones usando el lenguaje de programación Rust.
\end{itemize}

Desde el punto de vista académico, el proyectista adquiere las siguientes competencias y habilidades:

\begin{itemize}
    \item Conocimiento de la tecnología de captura de movimiento y su aplicación en el ámbito deportivo.
    \item Manejo del lenguaje de programación Rust.
    \item Entendimiento de las implicaciones de la programación multiplataforma, que requiere un conocimiento profundo de los sistemas operativos y el entorno web.
\end{itemize}

\section{Estructura de la memoria}

En los próximos capítulos de la memoria se ofrece información sobre el estado actual de la tecnología de captura de movimiento y se valoran las características necesarias con las que debe contar este proyecto, valorando las mejores herramientas para su desarrollo. A continuación, se presenta el diseño de la aplicación y su implementación. Por último, se valora el impacto del proyecto, así como su posible colaboración con los \ac{ODS}.