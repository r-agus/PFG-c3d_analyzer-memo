\chapter{Conclusiones y trabajo futuro} \label{sec:cap6}
% \noindent En el último apartado de la memoria se ha de incluir una visión general del trabajo realizado: problema propuesto, solución planteada y resultados obtenidos. Junto con esta descripción, también hay que especificar qué conocimiento nuevo se puede derivar de toda la información expuesta en el informe: validez o no del sistema, nuevos aspectos del problema detectados en el trabajo, etc. Finalmente, si el proyecto no deja totalmente cerrado y resuelto el problema tecnológico abordado, el proyectista debe aprovechar su experiencia para proponer líneas futuras de trabajo.

\noindent En este capítulo se presenta una visión general del trabajo realizado, incluyendo el problema propuesto, la solución planteada y los resultados obtenidos. Además, se discuten las implicaciones sociales, éticas y medioambientales del proyecto, así como su posible contribución a los Objetivos de Desarrollo Sostenible (ODS) de la ONU. Finalmente, se proponen líneas futuras de trabajo que podrían derivarse de este proyecto.

\section{Conclusiones}

Este proyecto ha permitido desarrollar una aplicación multiplataforma para la visualización de datos biomecánicos en un entorno tridimensional, utilizando el lenguaje de programación Rust. La aplicación es capaz de capturar y analizar datos de movimiento, proporcionando información valiosa para el estudio del rendimiento deportivo y la prevención de lesiones.

El lenguaje de programación Rust ha demostrado ser una herramienta capaz de ofrecer un alto rendimiento sin extender excesivamente el tiempo de desarrollo. Su gestor de paquetes, Cargo, en combinación con herramientas de control de versiones y un repositorio en GitHub, ha facilitado el desarrollo de la aplicación en múltiples entornos, con una migración sencilla entre sistemas operativos. Por otro lado, la aplicación ha sido diseñada para ser multiplataforma, lo que permite su uso en diferentes sistemas operativos (Windows, Linux y MacOS) y en entornos web.

La aplicación demuestra potencial para ser utilizada en diferentes ámbitos, poniendo foco en el análisis del movimiento al realizar un ejercicio físico, estando en consonancia con diferentes ODS de la ONU. En particular, el proyecto contribuye a la promoción de un estilo de vida saludable y activo (ODS 3), a la educación inclusiva y equitativa (ODS 4) y a la industrialización inclusiva y sostenible (ODS 9).

\section{Trabajos futuros}

Durante la realización de este proyecto se han sentado las bases para el desarrollo de una gran aplicación para el análisis del movimiento humano. Este proyecto se ha focalizado en el estudio del movimiento a partir de un \textit{swing} de golf, pero la aplicación debe ser capaz de analizar cualquier tipo de movimiento. 

Como línea futura, se propone probar la aplicación en diferentes entornos, no solo en el ámbito deportivo, sino también en el ámbito médico y de rehabilitación. Por otro lado, se sugiere la posibilidad de trabajar con diferentes formatos de ficheros, ademas de los ficheros \ac{C3D} usados en este proyecto. En este sentido, se podría trabajar con el formato propietario de optitrack, \textit{.tak}, o con otros formatos no binarios.

También se podría considerar la posibilidad de capturar datos en tiempo real, usando el flujo de datos del sistema de captura, añadiendo funcionalidades como la posibilidad de grabar el movimiento.

Por último, sería interesante añadir funcionalidades que hasta ahora requieren de un software externo, como la posibilidad de añadir marcadores al movimiento, o editar los que ya existen.
