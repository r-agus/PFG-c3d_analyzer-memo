\newpage
\section{Generación de vectores} \label{apx:spawn_ref_vec}

En este apéndice se presenta el código utilizado para la generación de vectores tridimensionales en el entorno visual. \textit{Bevy}, como motor de videojuegos, proporciona figuras geométricas primitivas como cilindros y conos, pero no ofrece una implementación directa para representar vectores direccionales, dado que su enfoque prioriza la ligereza antes que un paradigma totalmente funcional con todas las abstracciones preincorporadas.

La implementación mostrada a continuación combina estas primitivas geométricas (un cilindro para el cuerpo y un cono para la punta) para crear una representación visual de vectores. El código aprovecha características avanzadas de Rust como iteradores, closures y manejo de tipos de datos algebraicos, demostrando la expresividad y seguridad del lenguaje. Además, ilustra cómo extender las funcionalidades de \textit{Bevy} para adaptarlas a las necesidades específicas del proyecto.

\begin{lstlisting}[style=mystyle, language=Rust]
    fn spawn_reference_vectors(
        commands: &mut Commands, 
        materials: &mut ResMut<Assets<StandardMaterial>>, 
        meshes: &mut ResMut<Assets<Mesh>>
    ) {
        // Vectors for reference
        let default_cylinder_height = 0.25;
        let mut cone_mesh = Mesh::from(Cone {
            radius: 0.025,
            height: 0.1,
        });
        let mut vector_mesh = Mesh::from(Cylinder::new(
            0.01,
            default_cylinder_height,
        ));
    
        // Extract and modify positions
        if let Some(positions) = cone_mesh.attribute(Mesh::ATTRIBUTE_POSITION) {
            let modified_positions: Vec<[f32; 3]> = positions
                .as_float3()
                .unwrap_or(&[[0.0, 0.0, 0.0]])
                .iter()
                .map(|&[x, y, z]| [x, y + default_cylinder_height/2.0, z]) // cylinder height / 2, to place the cone on top of the cylinder (0 is the center of the cylinder)
                .collect();
            // Replace the positions attribute
            cone_mesh.insert_attribute(Mesh::ATTRIBUTE_POSITION, modified_positions);
        }
    
        vector_mesh.merge(&cone_mesh);
        
        let colors = vec![
            Color::srgb_u8(255, 0, 0),
            Color::srgb_u8(0, 255, 0),
            Color::srgb_u8(0, 0, 255),
        ];
        let point = Vec3::new(-2.0, -2.0, 0.0);
        let positions = vec![
            point + Vec3::new(default_cylinder_height / 2.0, 0.0, 0.0),
            point + Vec3::new(0.0, default_cylinder_height / 2.0, 0.0),
            point + Vec3::new(0.0, 0.0, default_cylinder_height / 2.0),
        ];
        let rotations = vec![
            Quat::from_axis_angle(Vec3::Z, -std::f32::consts::FRAC_PI_2),
            Quat::IDENTITY,
            Quat::from_axis_angle(Vec3::X, std::f32::consts::FRAC_PI_2),
        ];
        colors
            .iter()
            .zip(rotations)
            .zip(positions)
            .all(|((color, rotation), position)| {
            commands.spawn((
                Mesh3d(meshes.add(vector_mesh.clone())),
                MeshMaterial3d(materials.add(StandardMaterial {
                    base_color: color.clone(),
                    ..default()
                })),
                Transform::from_translation(position)
                    .with_rotation(rotation),
            ));
            true
        });
    }
\end{lstlisting}