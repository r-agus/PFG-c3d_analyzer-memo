\chapter{Preguntas Frecuentes}

\noindent Se describen a continuación las preguntas frecuentes que pueden surgir al utilizar la aplicación. Estas preguntas se han recopilado de los comentarios y sugerencias de los usuarios durante el desarrollo del proyecto. Se espera que estas preguntas y respuestas sean útiles para los usuarios y ayuden a resolver dudas comunes.

% \renewcommand{\thesection}{Q\arabic{section}}

\section{¿Puedo usar la aplicación en mi ordenador?}

En este momento, la aplicación es propiedad intelectual de la \ac{UPM} y no está disponible para su uso público. La aplicación está diseñada para ser utilizada por profesores e investigadores de la \ac{UPM} y no está destinada a su distribución pública.

Sin embargo, si cumples con los requisitos de uso y deseas utilizar la aplicación, puedes ponerte en contacto con el grupo de investigación \ac{GAMMA} o con el departamento de \ac{INEF} de la \ac{UPM} para obtener más información sobre cómo acceder a la aplicación y su uso.

\section{¿Qué recursos hardware necesito para usar la aplicación?}

La aplicación está diseñada para funcionar en una amplia variedad de sistemas operativos y hardware. El proyecto se ha probado exhaustivamente en el sistema operativo Windows 11, y se han realizado pruebas en Linux (Fedora) y Mac OS. Sin embargo, no se garantiza que la aplicación funcione en todas las versiones de estos sistemas operativos. Se recomienda disponer de un ordenador con una tarjeta gráfica dedicada, y al menos 8 GB de memoria RAM.

\section{Quiero contribuir al proyecto, ¿cómo puedo hacerlo?}

Para dar continuidad al proyecto, tan solo es necesario disponer de una instalación completa de Rust. Se recomienda estar familiarizado con este lenguaje de programación, así como con el uso de Git.

\section{¿Qué diferencias hay entre la versión web y la versión de escritorio?}

La versión web de la aplicación está diseñada para ser utilizada en navegadores web y no requiere instalación. La versión de escritorio, por otro lado, se instala en el ordenador y ofrece una experiencia de usuario más completa y fluida. Ambas versiones tienen funcionalidades similares, pero la versión de escritorio puede ofrecer un rendimiento mejorado.

\section{¿En qué circunstancia debería usar la versión web?}

No se recomienda el uso de la versión web, ya que la versión de escritorio ofrece un rendimiento mejorado. Sin embargo, si no se puede instalar la versión de escritorio, la versión web es una alternativa viable.

\section{¿La aplicación está disponible en versión móvil?}

La aplicación no es dependiente de la plataforma, lo que significa que teóricamente podría ejecutarse en dispositivos móviles. Sin embargo, para el manejo de la cámara y navegación por el espacio 3D, es necesario el uso de un ratón. Además, existen muchos atajos de teclado que mejoran la experiencia. Por lo tanto, se podría usar en entornos móviles siempre que se disponga de un ratón y teclado. Se debe tener en cuenta que la aplicación es exigente en cuanto a recursos, por lo que no se garantiza una experiencia fluida. 
