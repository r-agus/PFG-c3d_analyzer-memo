\chapter{Herramientas utilizadas}

\noindent En este proyecto se han utilizado diversas herramientas y tecnologías para el desarrollo de la aplicación. A continuación, se detallan las principales herramientas utilizadas:

\section{Lenguajes de programación}

\subsection{Rust}

Rust es un lenguaje de programación moderno, enfocado en la seguridad y el rendimiento. Su sistema de tipos y su modelo de propiedad permiten evitar errores comunes en otros lenguajes, como los errores de memoria. Rust es un lenguaje compilado, lo que significa que el código se traduce a código máquina antes de ejecutarse, lo que mejora el rendimiento.

Rust ha sido el lenguaje principal utilizado en este proyecto, ya que permite el desarrollo de aplicaciones complejas y de alto rendimiento. Su ecosistema de bibliotecas y herramientas, como Cargo, facilita la gestión de dependencias y la construcción de proyectos.

\subsubsection{\textit{Crates} utilizados}

Un \textit{crate} es la unidad mínima de compilación en Rust. Un \textit{crate} puede ser una biblioteca o un ejecutable. En este proyecto se han utilizado varios \textit{crates} para facilitar el desarrollo de la aplicación. En este proyecto se han utilizado los siguientes \textit{crates}:

\begin{itemize}
    \item \textbf{\textit{Bevy}:} Es el motor de videojuegos utilizado en este proyecto. \textit{Bevy} es un motor de código abierto \autocite{BevyEngine}, bajo las licencias MIT \autocite{BevywebsiteLICENSEMain} y Apache 2.0 \autocite{BevyLICENSEAPACHEMain}.
    \item \textbf{\textit{c3dio}:} Es un \textit{crate} utilizado para el parsing de ficheros \ac{C3D}. Este \textit{crate} se ha utilizado como base, sobre el cual se ha desarrollado un \textit{parser} propio para el proyecto \autocite{C3dioRust}. Este crate está bajo la licencia MIT y Apache 2.0.
    \item \textbf{\textit{bevy\_c3d}:} Este \textit{crate} es una adaptación del \textit{crate} \textit{c3dio} para su uso en el motor de videojuegos \textit{Bevy}. Se ha modificado para adaptarse al resto del proyecto \autocite{BiomechanicsfoundationBevy_c3d2024}. Este \textit{crate} está bajo la licencia MIT y Apache 2.0.
    \item \textbf{\textit{Egui}:} Es una biblioteca para la generación de interfaces gráficas de usuario en Rust. Este \textit{crate} no se ha utilizado directamente, sino que se ha utilizado a través de la adaptación para \textit{Bevy}, llamada \textit{Bevy\_egui} \autocite{Bevy_eguiRust,ernerfeldtEmilkEgui2025}. Ambos \textit{crates} están bajo la licencia MIT y Apache 2.0.
    \item \textbf{\textit{Bevy\_egui}:} Es una adaptación de \textit{Egui} para su uso en \textit{Bevy}, permitiendo la creación de interfaces gráficas de usuario de manera más integrada en el motor. Como ya se ha mencionado, este \textit{crate} está bajo la licencia MIT y Apache 2.0.
    \item \textbf{\textit{egui\_plot}:} Es una biblioteca para la creación de gráficos interactivos en \textit{Egui}. Este \textit{crate} permite visualizar datos de manera efectiva y atractiva \autocite{ernerfeldtEmilkEgui_plot2025}. Al igual que los anteriores, este \textit{crate} está bajo la licencia MIT y Apache 2.0.
    \item \textbf{\textit{egui\_double\_slider}:} Es un \textit{crate} que permite la creación de deslizadores dobles en \textit{Egui}. Este \textit{crate} se ha utilizado para facilitar la selección de rangos de valores en la interfaz gráfica de usuario \autocite{hacknusHacknusEgui_double_slider2025}. Este \textit{crate} está bajo la licencia Apache 2.0.
    \item \textbf{\textit{bevy\_web\_file\_drop}:} Es un \textit{crate} que facilita la captura de archivos arrastrados y soltados en entornos web. Este \textit{crate} se encuentra bajo la licencia MIT y Apache 2.0 \autocite{kayhKayhhhBevy_web_file_drop2024}.
    \item \textbf{\textit{regex}:} Es una biblioteca para el manejo de expresiones regulares en Rust. Este \textit{crate} se utiliza para detectar los marcadores, uniones o vectores que se deben representar \autocite{RegexRust}. Este \textit{crate} está bajo la licencia MIT y Apache 2.0.
    \item \textbf{\textit{serde}:} Es una biblioteca para la serialización y deserialización de datos en Rust. Este \textit{crate} se utiliza para convertir estructuras de datos de Rust a formatos como JSON y viceversa \autocite{SerdeCratesioRust2025}. Este \textit{crate} está bajo la licencia MIT y Apache 2.0. 
    \item \textbf{\textit{toml}:} Es una biblioteca para el manejo de archivos TOML en Rust. Este \textit{crate} se utiliza para leer y escribir archivos de configuración en formato TOML \autocite{TomlCratesioRust2025}. Este \textit{crate} está bajo la licencia MIT y Apache 2.0.
    \item \textbf{\textit{base64}:} Es una biblioteca para la codificación y decodificación de datos en formato Base64. Este \textit{crate} se utiliza para codificar y decodificar datos binarios en cadenas de texto. Este \textit{crate} está bajo la licencia MIT y Apache 2.0 \autocite{Base64CratesioRust2024}.
    \item \textbf{\textit{js-sys}:} Es una biblioteca que proporciona enlaces a las API de JavaScript desde Rust. Este \textit{crate} se utiliza para interactuar con el entorno de ejecución de JavaScript en el navegador \autocite{JssysCratesioRust2025}. Este \textit{crate} está bajo la licencia MIT y Apache 2.0.
    \item \textbf{\textit{wasm-bindgen}:} Es una biblioteca que facilita la interoperabilidad entre Rust y JavaScript. Este \textit{crate} se utiliza para generar enlaces entre el código Rust y el código JavaScript, permitiendo la comunicación entre ambos. Este \textit{crate} está bajo la licencia MIT y Apache 2.0 \autocite{IntroductionWasmbindgenGuide,RustwasmWasmbindgen2025}.
    \item \textbf{\textit{winit}:} Es una biblioteca para la creación de ventanas y la gestión de eventos en aplicaciones de escritorio. Este \textit{crate} se utiliza para crear la ventana principal de la aplicación y gestionar los eventos del usuario. Este \textit{crate} está bajo la licencia Apache 2.0 \autocite{RustwindowingWinit2025}. 
\end{itemize}

\subsection{JavaScript, HTML y CSS}

JavaScript es un lenguaje de programación utilizado principalmente para el desarrollo web. En este proyecto, se ha utilizado JavaScript para la creación de un entorno básico, sobre el cual se ejecuta la aplicación desarrollada en Rust. Hay una interacción mínima entre el código JavaScript y el código Rust: cuando la aplicación ha terminado de cargar, se envía un mensaje al código JavaScript para que elimine la pantalla de carga. Para dicha pantalla de carga, se ha utilizado CSS y HTML. El CSS se ha utilizado para dar estilo a la pantalla de carga, mientras que el HTML se ha utilizado para estructurar el contenido de la pantalla de carga.

\subsection{Python}

Python es un lenguaje de programación de alto nivel, interpretado y de propósito general. Gracias a la gran cantidad de bibliotecas y herramientas disponibles, es fácil crear prototipos y pruebas de concepto. En este proyecto se ha utilizado Python para comprobar el funcionamiento del \textit{parser} de los ficheros \ac{C3D}. Se ha utilizado la biblioteca \textit{ezc3d} \autocite{PyomecaEzc3d2025} y \textit{py-c3d} \autocite{EmbodiedCognitionPyc3d2025}. También se ha utilizado Python para la generación de las tablas de esta memoria.

\section{Control de versiones}

En este proyecto se ha utilizado Git como sistema de control de versiones, utilizando un repositorio privado en GitHub como plataforma de alojamiento del código, donde se han registrado más de 300 \textit{commits} y más de 25 \textit{pull requests}. Además, se han utilizado herramientas de integración continua como GitHub Actions para automatizar el proceso de pruebas, y GitHub Pages para alojar la versión web de la aplicación.

\section{Entornos de desarrollo}

Como entornos de desarrollo se han utilizado Visual Studio Code y Zed. Ambos entornos son ligeros y permiten una personalización avanzada, lo que facilita el trabajo con Rust y otros lenguajes de programación. Además, se han utilizado extensiones específicas para Rust, como \textit{rust-analyzer} y \textit{rustfmt}, para mejorar la experiencia de desarrollo.

\section{Otras herramientas}
Se han utilizado las siguientes herramientas adicionales para el desarrollo y la gestión del proyecto:

\begin{itemize}
    \item \textbf{LaTeX:} Para la redacción de la documentación y la memoria del proyecto.
    \item \textbf{Zotero:} Para la gestión de referencias bibliográficas.
    \item \textbf{PlantUML:} Para la creación de diagramas y visualizaciones.
\end{itemize}

\section{Ordenadores, sistemas operativos y navegadores utilizados}

Para el desarrollo de este proyecto se ha utilizado principalmente un ordenador portátil LG Gram 14, con un procesador Intel Core i7-1260P y 16 GB de RAM. Este ordenador ha sido utilizado para el desarrollo de la aplicación, así como para la redacción de la memoria y la gestión del proyecto. Además, se han utilizado otros ordenadores para realizar pruebas y verificar el funcionamiento de la aplicación en diferentes entornos. Entre ellos se encuentra un Lenovo V15 G3 con un procesador Intel Core i3-1215U y 8 GB de RAM, un Macbook Pro M1 de 13 pulgadas y ordenador personalizado con un procesador AMD Ryzen 7 5800X y 32 GB de RAM, utilizando el sistema operativo Windows 11 y Fedora 38. El desarrollo se ha realizado principalmente en el sistema operativo Windows 11, aunque también se han realizado pruebas en Fedora 38 y MacOS Ventura.

Como navegadores, se ha utilizado Microsoft Edge (basado en Chromium), Google Chrome, Mozilla Firefox y Safari.